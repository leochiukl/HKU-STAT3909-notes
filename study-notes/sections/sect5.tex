\section{Profit Analysis}
\label{sect:profit-analysis}
\begin{enumerate}
\item The profit analysis in \Cref{sect:profit-analysis} can be divided into
two types:
\begin{enumerate}[label={(\arabic*)}]
\item \emph{Classifying profits by sources:} While the profit amount can be
obtained by a simple ``\(\text{revenue}-\text{cost}\)'', the insurer would also
like to know the \emph{contributions} to the profits from different factors,
e.g. interest and mortality. Here we will discuss a method to do that.

\item \emph{(Main focus)} \emph{Testing the profits under different scenarios:}
Also known as \defn{profit testing}, here we are investigating the impacts on
the evolution of future net cash flows and emergence of profits upon changes in
assumptions made (e.g., interest and mortality assumptions). Profit testing is
an application of \emph{scenario analysis} (learnt in STAT3904) in a life
contingencies context.

While the primary purpose of profit testing is to determine the profitability
of various products offered by the insurer, it can actually help us to perform
core actuarial functions, namely \emph{pricing} and \emph{reserving}, which
raises actuaries' interest in this topic.
\end{enumerate}
\begin{note}
Throughout \Cref{sect:profit-analysis}, we shall focus on discussing the
profit analysis of insurance products.
\end{note}
\end{enumerate}
\subsection{Classification of Profits by Sources}
\label{subsect:classify-profits}
\begin{enumerate}
\item Recall the basic policy value recursion formula we learn from STAT3901:
\[
\underbrace{(\Vx[t]{}+G_t-e_t)(1+i)}_{\text{what you have}}=
\underbrace{\qx{x+t}(S_{t+1}+E_{t+1})+\px{x+t}(\Vx[t+1]{}[])}_{\text{what you need (expected)}}
\]
(\(t\) is an integer time). Intuitively, it can be interpreted as saying that
``what you have is what you \emph{expect to} need''. In practice, ``what
you \emph{actually} need'' \faIcon{at} time \(t+1\) would almost always
differ from your expectation, like many things in real life. In fact, since
``what you have'' is calculated based on assumptions (e.g., interest rate),
even ``what you \emph{actually} have'' could be different from what you
calculated, when your assumptions do not match exactly with the actual
experience!

The difference \(\text{``what you \emph{actually} have''}-\text{``what you
\emph{actually} need''}\) is indeed the \emph{actual} profit from
the policy; It is good \faIcon[regular]{thumbs-up} for ``what you
\emph{actually} have'' to exceed ``what you \emph{actually} need'',
as you can keep some amount of ``what you have'' as profit
\faIcon[regular]{grin-beam}.  On the other hand, it is bad
\faIcon[regular]{thumbs-down} for ``what you \emph{actually} need'' to
exceed ``what you \emph{actually} have'', since it means that you have no
choice but to use ``your own money'' to cover the shortfall, resulting in a
loss \faIcon[regular]{sad-tear}!

\item \textbf{Notations.} Let us now introduce some notations related to profit
calculations:
\begin{center}
\begin{tabular}{ll}
\toprule
Notation&Meaning\\
\midrule
\(\lx{x+t}[\mathrm{actual}]\)&actual number of ``survivors'' \faIcon{at} time \(t\) \\
\(i_{t}^{\mathrm{actual}}\)&actual annual effective interest rate \faIcon{at} time \(t\) \\
\(e_t^{\mathrm{actual}}\)&sum of actual initial and renewal expenses (per policy) \faIcon{at} time \(t\)\\
\(E_{t+1}^{\mathrm{actual}}\)&actual settlement expense (per policy) \faIcon{at} time \(t+1\)\\
\bottomrule
\end{tabular}
\end{center}
\begin{remark}
\item Like the asset shares case in \Cref{subsect:asset-shares}, ``survivors''
actually mean policies in force, or ``surviving policies''. But sometimes we
still use the term ``survivors'' to simplify wordings.
\item For other types of quantities like sum insured and premiums, we assume
that they are fixed throughout the whole policy term and do not depend on the
actual experience.
\end{remark}

\item \textbf{Total amount of profits.} Before performing the classification,
we need to first calculate the total amount of profits, which is given by:
\[
\lx{x+t}[\vc{\mathrm{actual}}]\bigg(
\underbrace{(\Vx[t]{}+G_t-e_t^{\vc{\mathrm{actual}}})(1+i_t^{\vc{\mathrm{actual}}})}_{\text{what you \vc{actually} have}}
-\underbrace{\qx{x+t}^{\vc{\mathrm{actual}}}(S_{t+1}+E_{t+1}^{\vc{\mathrm{actual}}})
+\px{x+t}^{\vc{\mathrm{actual}}}(\Vx[t+1]{}[])}_{\text{what you \vc{actually} need}}
\bigg)
\]
where
\(\px{x+t}^{\vc{\mathrm{actual}}}=\lx{x+t+1}[\vc{\mathrm{actual}}]/\lx{x+t}[\vc{\mathrm{actual}}]\)
and \(\qx{x+t}^{\vc{\mathrm{actual}}}=1-\px{x+t}^{\vc{\mathrm{actual}}}\).

\item\label{it:classify-profits} \textbf{Classifying profits by sources.} Now
we introduce one method to decompose the total profit into several components,
each for one source.  The key idea is as follows:
\[
\text{Profit due to \(\square\)}
=\text{Profit with actual \(\square\)}
-\text{Profit with assumed \(\square\)}
\]
where \(\square\) stands for ``interest'', ``mortality'', or ``expense''.

More specifically, we are decomposing the total profit as follows:
\begin{align*}
\text{Total profit}&=\text{Profit with all actual quantities}-\overbrace{\text{Profit with all assumed quantities}}^{\text{always \(0\)}} \\
&=\text{Profit with actual interest, expenses \& mortality}\vc{-\text{Profit with actual interest \& expenses}} \\
&\quad\vc{+\text{Profit with actual interest \& expenses}}\gc{-\text{Profit with actual interest}} \\
&\quad\gc{+\text{Profit with actual interest}}-\text{Profit with all assumed quantities} \\
&=\text{Profit due to mortality} \\
&\quad+\text{Profit due to expenses} \\
&\quad+\text{Profit due to interest}.
\end{align*}
While this decomposition is intuitively appealing, it actually has one major
limitation \warn{}, namely that the profits due to various sources obtained
\emph{depend on the ``order'' of classifications}. In the case above, the
``order'' would be
\[
\text{Interest}\to\text{Expenses}\to\text{Mortality}
\]
which refers to the order for converting the assumed quantity to the actual
one, starting from the ``all assumed quantities'' case:
\begin{itemize}
\item \(\text{Profit due to interest}=\text{Profit with actual interest}-\text{Profit with all assumed quantities}\) \\
(we convert assumed interest to actual interest here).
\item \(\text{Profit due to expenses}=\text{Profit with actual interest \& expenses}
-\text{Profit with actual interest}\) \\
(we convert assumed expenses to actual expenses here).
\begin{warning}
Make sure you consider \underline{all} kinds of expenses: initial, renewal, and
settlement expenses.
\end{warning}
\item \(\text{Profit due to mortality}=\text{Profit with actual int., exp. \& mortality}
-\text{Profit with actual int. \& exp.}\) \\
(we convert assumed mortality to actual mortality here).
\end{itemize}

The values obtained could be different if we decompose the profits in the
following order instead
(\(\text{Mortality}\to\text{Expenses}\to\text{Interest}\)):
\begin{align*}
\text{Total profit}&=\text{Profit with actual interest, expenses \& mortality}-\text{Profit with actual \rc{mortality} \& expenses} \\
&\quad+\text{Profit with actual \rc{mortality} \& expenses}-\text{Profit with actual \rc{mortality}} \\
&\quad+\text{Profit with actual \rc{mortality}}-\text{Profit with all assumed quantities}.
\end{align*}
\item \textbf{Shortcuts for decomposing profits.} Some shortcuts are available
when we decompose the profits as above, due to cancellations of terms. To
illustrate them, let us use the decomposition in the order
\(\text{Interest}\to\text{Expenses}\to\text{Mortality}\) as an example:
\begin{itemize}
\item \emph{Profit due to interest:}
\begin{align*}
&\text{Profit with actual interest}-\overbrace{\text{Profit with all assumed quantities}}^{0} \\
&=\quad\lx{x+t}[\mathrm{actual}]\bigg(
(\Vx[t]{}+G_t-e_t)(1+i_t^{\vc{\mathrm{actual}}})
-\qx{x+t}(S_{t+1}+E_{t+1})
+\px{x+t}(\Vx[t+1]{}[])\bigg).
\end{align*}

\item \emph{Profit due to expenses:}
\begin{align*}
&\text{Profit with actual interest \& expenses}
-\text{Profit with actual interest} \\
&=\quad\lx{x+t}[\mathrm{actual}]\bigg(
(\Vx[t]{}+G_t-e_t^{\vc{\mathrm{actual}}})(1+i_t^{\vc{\mathrm{actual}}})
-\qx{x+t}(S_{t+1}+E_{t+1}^{\vc{\mathrm{actual}}})
+\px{x+t}(\Vx[t+1]{}[])\bigg) \\
&\quad-\lx{x+t}[\mathrm{actual}]\bigg(
(\Vx[t]{}+G_t-e_t)(1+i_t^{\vc{\mathrm{actual}}})
-\qx{x+t}(S_{t+1}+E_{t+1})
+\px{x+t}(\Vx[t+1]{}[])\bigg) \\
&=\quad\lx{x+t}[\mathrm{actual}]\bigg[-(e_{t}^{\mathrm{actual}}-e_{t})(1+i_{t}^{\mathrm{actual}})
-\qx{x+t}(E_{t+1}^{\mathrm{actual}}-E_{t+1})\bigg].
\end{align*}

\item \emph{Profit due to mortality:}
\begin{align*}
&\text{Profit with actual interest,  expenses \& mortality}
-\text{Profit with actual interest \& expenses} \\
&\overset{\text{(NAAR form)}}{=}\lx{x+t}[\mathrm{actual}]\bigg(
(\Vx[t]{}+G_t-e_t^{\vc{\mathrm{actual}}})(1+i_t^{\vc{\mathrm{actual}}})
-\qx{x+t}[\vc{\mathrm{actual}}](S_{t+1}+E_{t+1}^{\vc{\mathrm{actual}}}-\Vx[t+1]{}[])
+\Vx[t+1]{}[]\bigg) \\
&\hspace{1.5cm}-\lx{x+t}[\mathrm{actual}]\bigg(
(\Vx[t]{}+G_t-e_t^{\vc{\mathrm{actual}}})(1+i_t^{\vc{\mathrm{actual}}})
-\qx{x+t}(S_{t+1}+E_{t+1}^{\vc{\mathrm{actual}}}-\Vx[t+1]{}[])
+\Vx[t+1]{}[]\bigg) \\
&\hspace{0.8cm}=\quad-\lx{x+t}[\mathrm{actual}](\qx{x+t}[\vc{\mathrm{actual}}]-\qx{x+t})(S_{t+1}+E_{t+1}^{\vc{\mathrm{actual}}}-\Vx[t+1]{}[]).
\end{align*}
\end{itemize}
\begin{note}
Here we always multiply the per-policy profit by the actual number of policies
\faIcon{at} time \(t\): ``Actual'' or ``assumed'' mortality is actually
referring whether \(\px{x+t}[\mathrm{actual}]\) \&
\(\qx{x+t}[\mathrm{actual}]\) or \(\px{x+t}\) \& \(\qx{x+t}\) should be used.
\end{note}
\end{enumerate}
\subsection{Profit Testing}
\begin{enumerate}
\item To conduct profit testing, we will make extensive use of the following
notations:
\begin{itemize}
\item \(E_0\): \defn{pre-contract expense}. This is given by initial expense
less renewal expense and is conventionally treated as a payment made ``at the
end of year 0''\footnote{This is in contrast with the first premium payment
time, which is at the beginning of year 1. Mathematically we can treat them as
referring to the same time point, but for some terms appearing later, it is
essential to distinguish them carefully.}.

\item \(e_{t-1}\): renewal expense made at the beginning of year \(t\) (i.e.,
time \(t-1\)). Note that the initial expense is decomposed into pre-contract
expense \(E_0\) and ``renewal expense'' \(e_0\) made at the beginning of year 1
here.

\item \(G_{t-1}\): gross premium payable at the beginning of year \(t\).

\item \(S_t\): death benefit payable at the end of year \(t\) if death occurs
in that year.
\item \(i_{\mathrm{acc}}\): annual effective interest rate (or ``accumulation
rate'') used in the profit testing.
\end{itemize}
Armed with these notations, we can now discuss several important quantities
involved in profit testing.
\item \label{it:exp-ncf-fmlas} \textbf{Expected net cash flows.} There are two
kinds of expected net cash flows, which can be distinguished by whether we are
considering \emph{per policy in-force} or \emph{per policy issued}.

\begin{itemize}
\item \emph{Per policy in-force (at the start of year \(t\)):} The
\defn{in-force expected net cash flow} (or \defn{in-force emerging surplus})
for year \(t\), denoted by \(CF_{t}\), is the expected net cash flow obtained
by accumulating (at \(i_{\mathrm{acc}}\)) all the (expected) cash flows in year
\(t\) to the year end:
\[
\boxed{CF_{0}=-E_0}\qqtext{and}
\boxed{CF_{t}=(G_{t-1}-e_{t-1})(1+i_{\mathrm{acc}})-S_t\qx{x+t-1}}\text{ for any \(t=1,2,\dotsc\)}.
\]
\begin{center}
\begin{tikzpicture}
\draw[-Latex] (0,0) -- (10,0) node[right]{Time};
\fill[] (3,0) circle [radius=0.05]
node[below] {\(t-1\)};
\fill[] (7,0) circle [radius=0.05]
node[below] {\(t\)};
\draw[->, ForestGreen, line width=0.3mm] (3,0.6) --node[midway, left]{\faIcon{plus-circle}} (3,0.2)
node[pos=-0.5]{\(G_{t-1}-e_{t-1}\)};
\draw[->, red, line width=0.3mm] (7,0.2) --node[midway, right]{\faIcon{minus-circle}} (7,0.6)
node[pos=1.8] {\(S_{t}\qx{x+t-1}\)};
\draw[-Latex, blue] (3,0.1) to[bend right] node[midway, auto, swap]{\(\times (1+i_{\mathrm{acc}})\)} (7,0.1);
\end{tikzpicture}
\end{center}
Here we have \(CF_0=-E_0\) since the only (negative) cash flow in ``year 0'' is
\(-E_0\), corresponding to the pre-contract expense ``at the end of year 0''.

\item \emph{Per policy issued:} The \defn{expected net cash flow per policy
issued} for year \(t\), denoted by \(EC_{t}\), is given by
\[
\boxed{EC_{0}=CF_{0}}\qqtext{and}
\boxed{EC_{t}}=\frac{\overbrace{\lx{x+t-1}CF_{t}}^{\mathclap{\text{total expected NCF}}}}
{\underbrace{\lx{x}}_{\mathclap{\text{no.\ of policies issued}}}}=\boxed{\px[t-1]{x}\times CF_{t}}
\text{ for any \(t=1,2,\dotsc\)}.
\]
\end{itemize}
\item \label{it:exp-profit-fmlas} \textbf{Expected profits.} Again there are
two kinds of expected profits, which can similarly distinguished by whether
\emph{per policy in-force} or \emph{per policy issued} is considered.
\begin{itemize}
\item \emph{Per policy in-force:} The \defn{in-force expected profit} for year
\(t\), denoted by \(PR_{t}\), is like the profit we have seen in
\Cref{subsect:classify-profits}, and can be expressed as \(\text{``what you
have''}-\text{``what you need''}\), with the latter being expected value:
\[
\boxed{PR_{0}=-\underset{\text{(what you need)}}{E_0}}\qqtext{and}
\boxed{PR_{t}=\underset{\text{(what you have)}}{(\Vx[t-1]{}+G_{t-1}-e_{t-1})(1+i_{\mathrm{acc}})}-
\underset{\text{(what you need)}}{\qx{x+t-1}(S_{t}+E_{t})+\px{x+t-1}(\Vx[t]{}[])}}
\]
for any \(t=1,2,\dotsc\). While it appears that \(PR_{t}\) is always zero for
any \(t=1,2,\dotsc\) due to the basic policy value recursion formula, it is
actually not the case \warn{}. The reason is that the profit testing
interest rate \(i_{\mathrm{acc}}\) may NOT be the same as the interest rate
used for computing policy values (the one appearing in the recursive formula).
\item \emph{Per policy issued:} The \defn{expected profit per policy issued}
for year \(t\), denoted by \(\sigma_t\), is given by
\[
\boxed{\sigma_0=PR_{0}}\qqtext{and}
\boxed{\sigma_{t}}=\frac{\overbrace{\lx{x+t-1}PR_{t}}^{\mathclap{\text{total expected profit}}}}
{\underbrace{\lx{x}}_{\mathclap{\text{no.\ of policies issued}}}}=\boxed{\px[t-1]{x}\times PR_{t}}
\text{ for any \(t=1,2,\dotsc\)}.
\]
\end{itemize}
After collecting the in-force expected profits and expected profits per policy
issued for different years into vectors, there are special names for the
vectors:
\begin{itemize}
\item \defn{Profit vector}: \((PR_{0},PR_{1},\dotsc,PR_{n})\).
\item \defn{Profit signature}: \((\sigma_{0},\sigma_{1},\dotsc,\sigma_{n})\).
\end{itemize}
\begin{note}
Here we suppose that a policy with \(n\) years term is being considered, so we
only include the values up to year \(n\).
\end{note}
\item \textbf{Relationship between expected net cash flows and profits.}
There is an interesting relationship between the in-force expected NCF \(CF_t\)
and the in-force expected profit \(PR_t\) as follows:
\[
PR_t=CF_t-\blc{\Vx[t-1]{}[](1+i_{\mathrm{acc}})-\px{x+t-1}(\Vx[t]{}[])}.
\]
It is not hard to verify this directly but checking the definitions (try
this!). The interesting thing about this relationship is that a rather
intuitive interpretation can be obtained by noting that the \blc{blue}
expression is the (expected) increase in reserve per policy in-force. Denoting
it by \(IR_{t}\), we can write \(\boxed{PR_t=CF_t-IR_{t}}\), which can be
intuitively understood as saying that:
\[
\text{profit you have}
=\underset{\text{(from the NCF)}}{\text{amount \faIcon{dollar-sign} \gc{going into} your pocket}}
-\underset{\text{(for increasing reserve)}}{\text{amount \faIcon{dollar-sign} \rc{going out of} your pocket}}.
\]
\begin{note}
To see why the \blc{blue} expression is the increase in reserve per policy
in-force, write
\[
\blc{\Vx[t-1]{}[](1+i_{\mathrm{acc}})-\px{x+t-1}(\Vx[t]{}[])}
=\frac{\overbrace{\lx{x+t-1}\times \Vx[t-1]{}[](1+i_{\mathrm{acc}})-\lx{x+t}\times \Vx[t]{}[]}^{\text{total (expected) increase in reserve}}}
{\underbrace{\lx{x+t-1}}_{\text{no.\ of policies in-force}}}
\]
where we note that the start-of-year reserve \(\Vx[t-1]{}[]\) can earn interest
at rate \(i_{\mathrm{acc}}\) in year \(t\), and at the end of year \(t\), the
reserve amount \(\Vx[t]{}[]\) is only for each policy in-force at that time.
\end{note}
\item \textbf{Profit measures.} After learning about these important quantities
involved in profit testing, it is time to actually apply them in profit
testing.  Here we will do that by using some \emph{profit measures} which, as
its name suggests, \underline{measure} \underline{profits}. In fact, you should
have seen some of them back in STAT3904. But we will go through all of them in
details here, in case you already forget \(\cancel{\text{\faIcon{brain}}}\) the
STAT3904 content (pretty normal, I guess).

Since our goal here is to measure \underline{profits} per policy
\underline{issued}\footnote{We want to analyze the profitability right at the
onset, to decide whether we should offer those products or not at the very
beginning.}, we will make extensive use of the expected profits \emph{per policy
issued} \(\sigma_t\)'s. Four profit measures will be discussed here:
\begin{itemize}
\item Net present value (NPV)
\item \emph{(new!)} Profit margin
\item Discounted payback period (DPP)
\item Internal rate of return (IRR)
\end{itemize}
\begin{note}
Henceforth we shall consider a policy with \(n\) years term.
\end{note}

\item \label{it:npv-fmla} \textbf{Net present value.}
\begin{enumerate}[label={(\arabic*)}]
\item \emph{Definition:} The \defn{net present value} (NPV) of a policy is the
sum of all discounted \(\sigma_t\)'s (\faIcon{at} an interest rate \(r\), known
as the \defn{required rate of return}, \defn{hurdle rate}, or \defn{risk
discount rate}):
\[
\boxed{\text{NPV \faIcon{at} \(r\)}=\sum_{t=0}^{n}\sigma_t(1+r)^{-t}}.
\]
\item \emph{Discussion:} We have the following relationships between how we
set reserves and the emerging pattern of expected profits:
\begin{itemize}
\item Setting higher reserves \gc{earlier} \faIcon{arrow-right} Higher \(\sigma_t\)'s emerging \rc{later}
\item Setting higher reserves \rc{later} \faIcon{arrow-right} Higher \(\sigma_t\)'s emerging \gc{earlier}
\end{itemize}
Often, the hurdle rate exceeds the profit testing interest rate, i.e.,
\(r>i_{\mathrm{acc}}\). In such case, higher \(\sigma_t\)'s emerging
\gc{earlier} leads to a higher NPV. Hence, setting higher reserves \rc{later}
would usually result in a higher NPV.
\end{enumerate}

\item \label{it:profit-marg-fmla} \textbf{Profit margin.}
The \defn{profit margin} of a policy \faIcon{at} the hurdle rate \(r\) is the
NPV per expected PV of premiums:
\[
\text{Profit margin \faIcon{at} \(r\)}
=\frac{\overset{\text{(total NPV)}}{\vc{\lx{x}}\times \text{NPV \faIcon{at} \(r\)}}}
{\underset{\text{(total expected PV of premiums)}}{\sum_{t=0}^{n-1}G_t\times \vc{\lx{x+t}}\times (1+r)^{-t}}}
=\boxed{\frac{\text{NPV \faIcon{at} \(r\)}}
{\sum_{t=0}^{n-1}G_t\times \vc{\px[t]{x}}\times (1+r)^{-t}}}.
\]
Intuitively, profit margin tells the proportion of premiums earned as profits.

\item\label{it:dpp-fmla} \textbf{Discounted payback period.} The
\defn{discounted payback period} (DPP) is the earliest time at which the NPV is
nonnegative:
\[
\text{DPP}=\min\{\mgc{k}\in\{0,1,2,\dotsc\}: \text{NPV}_{\mgc{k}}\ge 0\}
\]
where \(\text{NPV}_{\mgc{k}}=\sum_{t=0}^{\mgc{k}}\sigma_t(1+r)^{-t}\) is the
\defn{partial NPV up to time \(k\)}, for any \(k=0,1,\dotsc,n\).

DPP \emph{does not exist} when \(\text{NPV}_{k}<0\) for any
\(k=0,1,,\dotsc,n\). In such case, it means that the policy ``never pays back''
and thus it is probably not a good idea to offer such policy.

\item \label{it:irr-fmla} \textbf{Internal rate of return.} The \defn{internal
rate of return} (IRR) is any interest rate at which the NPV is zero, i.e., any
value \(i_{\mathrm{IRR}}\) such that \(\text{NPV \faIcon{at} \(i_{\mathrm{IRR}}\)}=0\).

\begin{note}
There can be no IRR or multiple IRRs. But in general, if the profit signature
\((\sigma_0,\sigma_1,\dotsc,\sigma_n)\) has exactly one sign change from
negative to positive, i.e., all negative terms before positive terms, then the
IRR is uniquely determined.
\end{note}

\item \textbf{Pricing by profit testing.} Now let us start discussing how we
can use profit testing to set premiums (pricing) and reserves (reserving). For
pricing, there can be many ways to utilize the results from profit testing. For
instance, the insurer may compute the NPV at a fixed hurdle rate for various
premiums, and set the premium as the \emph{breakeven premium}, i.e., the one
that leads to a zero NPV. In practice, it may be more feasible with the help of
a spreadsheet \faIcon{file-excel}.

\item \label{it:zeroization} \textbf{Reserving by profit testing.} The
discussion about using profit testing to set reserves is more interesting since
there is a special method to do that, which is known as \emph{zeroization}. As
you may expect, this method involves some ``zeros''.

\defn{Zeroization} is an iterative process which computes the reserves to be
set in a backward \faIcon{backward} fashion. Basically, zeroization determines
reserves that leads to zero profits in later years. Intuitively, this process
sets ``just enough'' reserves to avoid having excessively high reserves in
earlier years, which would lead to a lower NPV as suggested in
\labelcref{it:npv-fmla}.

More specifically, the zeroization process is as follows (for an \(n\)-year
policy):
\begin{itemize}
\item Solve \(PR_{n}=0\) for \(\Vx[n-1]{}[]\) \footnote{It
is possible since the boundary value \(\Vx[n]{}[]\) is known.} and label it as
\(\Vx[n-1]{}[Z]\).
\item Solve \(PR_{n-1}=0\) with
\(\Vx[n-1]{}[]\to\Vx[n-1]{}[Z]\) for \(\Vx[n-2]{}[]\), and label it as
\(\Vx[n-2]{}[Z]\).
\item \(\vdots\)
\item Solve \(PR_{1}=0\) with \(\Vx[1]{}[]\to\Vx[1]{}[Z]\) for
\(\Vx[0]{}[]\) and label it as \(\Vx[0]{}[Z]\).
\begin{warning}
The time-0 reserve is not necessarily zero!
\end{warning}
\end{itemize}
\begin{remark}
\item Often the formula \(PR_{t}=CF_{t}-IR_{t}\) is helpful here.
\item In case a \rc{negative} solution is obtained when solving \(PR_{t}=0\), we
should set the time-\(t\) reserve \(\Vx[t]{}[Z]\) as \(0\) (because it does not
make sense to have ``negative reserve''), and \(0\) is the smallest possible
reserve.
\item The value \(\Vx[t]{}[Z]\) is known as the time-\(t\) \defn{zeroized reserve}.
\end{remark}

\item \label{it:mult-state-profit-test-fmlas} \textbf{Profit testing for
multiple state model.} The last (nice!  \faIcon[regular]{smile-wink}) topic to
be discussed in \Cref{sect:profit-analysis} relates profit testing with what we
have learnt in \Cref{sect:mult-state-models}: multiple state model. Here we
will \emph{generalize} the quantities for profit testing from
\labelcref{it:exp-ncf-fmlas,it:exp-profit-fmlas}, and the generalized
quantities can be applied in profit testing through profit measures.

\begin{enumerate}[label={(\arabic*)}]
\item \emph{Notations:} In multiple state model, we would add superscripts to
the notations previously used (like what we have done in
\Cref{subsect:mult-state-prem-pv}), to indicate the states in which the
quantities are applicable:
\begin{itemize}
\item \emph{(pre-contract expense)} \(E_0\) remains unchanged
\item \emph{(renewal expense)} \(e_{t-1}\to e_{t-1}^{(j)}\)
\item \emph{(gross premium)} \(G_{t-1}\to G_{t-1}^{(j)}\)
\item \emph{(death benefit)} \(S_t\to S_t^{(j)}\)
\item \(i_{\mathrm{acc}}\) remains unchanged (assuming it is the same for every state)
\item \(CF_{t}\to CF_{t}^{(j)}\)
\item \(EC_{t}\) remains unchanged
\item \(PR_{t}\to PR_{t}^{(j)}\)
\item \(\sigma_{t}\) remains unchanged
\end{itemize}
\begin{note}
We assume the starting state is fixed and clear from context, so we do not add
superscripts to \(E_0\), \(EC_{t}\), and \(\sigma_t\) as they all only depend
on the starting state. Without loss of generality, we suppose the starting
state is \(0\) henceforth.
\end{note}
\item \emph{Expected net cash flow...}
\begin{itemize}
\item \emph{per policy in-force (in state \(j\)):}
\[
CF_{0}^{(0)}=-E_0\qqtext{and}
CF_{t}^{(j)}=(G_{t-1}^{(j)}-e_{t-1}^{(j)})(1+i_{\mathrm{acc}})-\sum_{k}^{}\px{x+t-1}[jk]S_t^{(k)}\text{ for any \(t=1,2,\dotsc\)}.
\]
\begin{center}
\begin{tikzpicture}
\draw[-Latex] (0,0) -- (10,0) node[right]{Time};
\fill[] (3,0) circle [radius=0.05]
node[below] {\(t-1\)};
\fill[] (7,0) circle [radius=0.05]
node[below] {\(t\)};
\draw[->, ForestGreen, line width=0.3mm] (3,0.6) --node[midway, left]{\faIcon{plus-circle}} (3,0.2)
node[pos=-0.5]{\(G_{t-1}^{(j)}-e_{t-1}^{(j)}\)};
\node[draw] (start) at (3,2) {State \(j\)};
\draw[->, red, line width=0.3mm] (7,0.2) --node[midway, right]{\faIcon{minus-circle}} (7,0.6);
\foreach \x in {0,1,2}{
\node[draw] () at (8.2,\fpeval{\x+1}) {State \x};
\draw[-Latex] (start) --node[midway, red]{\(\px{x+t-1}[j\x]\)} (6.7,\fpeval{\x+1});
\node[red] () at (7,\fpeval{\x+1}) {\(S_t^{(\x)}\)};
}
\draw[-Latex, blue] (3,0.1) to[bend right] node[midway, auto, swap]{\(\times (1+i_{\mathrm{acc}})\)} (7,0.1);
\end{tikzpicture}
\end{center}
\item \emph{per policy issued:}

\[
EC_{0}=CF_{0}^{(0)}\qqtext{and}
EC_{t}=\frac{\overbrace{\sum_{j}^{}\lx{x+t-1}^{(j)}CF_{t}^{(j)}}^{\mathclap{\text{total expected NCF}}}}
{\underbrace{\lx{x}^{(0)}}_{\mathclap{\text{no.\ of policies issued}}}}
=\sum_{j}^{}\px[t-1]{x}[0j]\times CF_{t}^{(j)}
\text{ for any \(t=1,2,\dotsc\)}.
\]
\begin{note}
Here \(\lx{x+t-1}^{(j)}\) denotes the expected number of policies in-force in
state \(j\) \faIcon{at} time \(t-1\).
\end{note}
\end{itemize}
\item \emph{(more important) Expected profit...}
\begin{itemize}
\item \emph{per policy in-force (in state \(j\)):}
\begin{align*}
&PR_{0}^{(0)}=-\underset{\text{(what you need)}}{E_0}\qqtext{and} \\
&\boxed{PR_{t}^{(j)}=\underset{\text{(what you have)}}{(\Vx[t-1]{}[(j)]+G_{t-1}^{(j)}-e_{t-1}^{(j)})(1+i_{\mathrm{acc}})}-
\underset{\text{(what you need)}}{\sum_{k}^{}\px{x+t-1}[jk]\qty(B_{t}^{(k)}+E_{t}^{(k)}+\Vx[t]{}[(k)])}}
\end{align*}
for any \(t=1,2,\dotsc\), where \(B_{t}^{(k)}\) and \(E_{t}^{(k)}\) carry the
meanings from \labelcref{it:mult-state-basic-policy-val-recurs}.
\item \emph{per policy issued:}
\[
\sigma_0=PR_{0}^{(0)}\qqtext{and}
\boxed{\sigma_{t}}=\frac{\overbrace{\sum_{j}^{}\lx{x+t-1}^{(j)}PR_{t}^{(j)}}^{\mathclap{\text{total expected profit}}}}
{\underbrace{\lx{x}^{(0)}}_{\mathclap{\text{no.\ of policies issued}}}}
=\boxed{\sum_{j}^{}\px[t-1]{x}[0j]\times PR_{t}^{(j)}}
\text{ for any \(t=1,2,\dotsc\)}.
\]
\end{itemize}
\end{enumerate}
\end{enumerate}

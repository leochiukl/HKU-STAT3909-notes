\section{Policy Values II}
\label{sect:policy-values}
\begin{enumerate}
\item Building upon the knowledge of policy values learnt in STAT3901 (see the
section titled ``Policy Values I'' in the
\href{https://leochiukl.github.io/files/stat3901-study-notes.pdf}{STAT3901
study notes}), here we will explore some further topics:
\begin{itemize}
\item \emph{Continuous policy value recursion:} For each of life insurance and
life annuity, we have explored the \emph{annual}, \emph{\(1/m\)thly}, and
\emph{continuous} cases in STAT3901. But for policy value recursions, we have
only discussed the \emph{annual} and \emph{\(1/m\)thly} cases. The continuous
case is to be discussed here.

\item \emph{Asset shares:} In STAT3901 we have learnt the concept of
\emph{retrospective} policy value, which can be interpreted as the share
available in the \emph{expected} ``pool of money'' \faIcon{at} time \(t\) per
\emph{expected} survivor \faIcon{at} time \(t\).  Here, we are going to work
with \emph{actual} quantities instead of those expected ones. In other words,
we consider the \underline{share} available in the \emph{actual} ``pool of
money'' (the \underline{asset}) \faIcon{at} time \(t\) per \emph{actual}
survivor \faIcon{at} time \(t\).

\item \emph{Practical issues:} Some practical issues about policy values will
be discussed, namely:
\begin{itemize}
\item \emph{dealing with policy alterations:} A policyholder may be given the
flexibility to \underline{alter} his policy, e.g., change from a term life
insurance to a whole life insurance. The concept of policy value will be used
in determining the suitable adjustments needed on the benefit amounts.
\item \emph{negative policy value:} While negative policy value appears to
suggest that there would be expected future \emph{gain}, prudent actuaries
would avoid assigning a positive value to the policy. We will study why that is
the case and some pitfalls of having negative policy value.
\end{itemize}
\end{itemize}
\end{enumerate}
\subsection{Continuous Policy Value Recursion: Thiele's Differential Equation}
\label{subsect:policy-val-cts-recurs}
\begin{enumerate}
\item Recall the \(1/m\)thly version of policy value recursion in the net
amount at risk (NAAR) form:
\[
\underset{\text{(what you have)}}{(\Vx[t]{}[]+G_t-e_t)(1+i)^{\frac{1}{m}}}
=\underset{\text{(what you need)}}
{\Vx[t+\frac{1}{m}]{}[]
+\qx[\frac{1}{m}]{x+t}\overset{\text{(NAAR)}}{\qty(S_{t+\frac{1}{m}}+E_{t+\frac{1}{m}}-\Vx[t+\frac{1}{m}]{}[])}
}
\]
(\(t=0,1/m,2/m,\dotsc\)).

\begin{center}
\begin{tikzpicture}
\draw[] (0,0) rectangle (2,1);
\node[] () at (1, 1.4) {\(\Vx[t]{}\)};
\draw[->, ForestGreen, line width=0.3mm] (0.6,-0.5) to (0.6,0.2);
\node[ForestGreen] () at (0.6,-0.8) {\(G_t\)};
\draw[->, red, line width=0.3mm] (1.4,0.2) to (1.4,-0.5);
\node[red] () at (1.4,-0.8) {\(e_t\)};
\draw[-Latex, blue] (1,0.5) to[bend left] node[midway, auto]{\(\times (1+i)^{1/m}\)} (7,0.5);
\draw[] (4,-0.5) rectangle (10,1);
\node[] () at (11.5,0.25) {(what you have)};
\node[] () at (7,0.25) {\((\Vx[t]{}+G_t-e_t)(1+i)^{1/m}\)};
\node[rotate=90] () at (7,-1) {\faIcon{equals}};
\draw[] (4,-3) rectangle (8,-1.5);
\node[] () at (11.5,-2.25) {(what you need)};
\node[] () at (6,-2.25) {\(\Vx[t+\frac{1}{m}]{}[]\)};
\draw[] (8,-3) rectangle (10,-1.5);
\node[text width=2cm, violet] () at (9.1,-2.25) {extra for \faIcon{skull} (expected)};
\end{tikzpicture}
\end{center}
We can then observe that changes in policy values are influenced by multiple
factors: (i) \blc{interest}, (ii) \gc{premiums} \& \rc{renewal expenses}, and (iii)
\vc{\(\text{NAAR}\times \text{death prob.}\)} (it reduces the amount in ``what
we have'' that can be ``sent'' to the policy value \(\Vx[t+\frac{1}{m}]{}[]\)).

\item \label{it:thieles-diffeq} Now, let us state the recursive formula for
policy values in the continuous case, which is known as the \defn{Thiele's
differential equation}:
\[
\boxed{\dv{}{t}\Vx[t]{}[]=\Vx[t]{}[]\delta_t+(G_t-e_t)
-\underset{\text{(NAAR)}}{(S_t+E_t-\Vx[t]{}[])}\mu_{x+t}}
\]
where:
\begin{itemize}
\item \(G_t\) and \(e_t\) denote the time-\(t\) \emph{rate} of premiums and
renewal expenses respectively;
\item \(S_t\) and \(E_t\) denote the time-\(t\) sum insured and settlement
expense respectively.
\end{itemize}
\begin{remark}
\item Here we are in the fully continuous case, and we assume that there is no
initial expense.
\item Loosely, the differential equation relates ``\(\Vx[t]{}[]\) and
\(\Vx[t+\dd{t}]{}[]\)'', hence it is a ``continuous recursion''.
\end{remark}

While we can find some ``familiar'' terms like \(G_t-e_t\) and
\(S_t+E_t-\Vx[t]{}[]\) (NAAR) in the equation, it is not very clear what
intuitive meaning this differential equation has. To understand it more
intuitively, let us integrate both sides from \(0\) to a positive value \(s\),
which gives
\[
\Vx[s]{}[]-\Vx[0]{}[]=\int_{0}^{s}\dv{}{t}\Vx[t]{}[]\dd{t}
=\int_{0}^{s}\qty[\Vx[t]{}[]\delta_t+(G_t-e_t)
-(S_t+E_t-\Vx[t]{}[])\mu_{x+t}]\dd{t}.
\]
Abusing notations a bit, we may write
\[
\Vx[s]{}[]=\Vx[0]{}[]
+\int_{0}^{s}\blc{\Vx[t]{}[]\delta_t\dd{t}}+(\gc{G_t}-\rc{e_t})\dd{t}
-\vc{(S_t+E_t-\Vx[t]{}[])\mu_{x+t}\dd{t}}.
\]
We can then gain some intuition out of it:
\begin{enumerate}[label={(\arabic*)}]
\item Starting (time \(0\)) reserve: \(\Vx[0]{}[]\).
\item In every ``infinitesimal'' time interval \([t,t+\dd{t}]\) between time
\(0\) and time \(s\):
\begin{itemize}
\item \faIcon{plus-circle} \emph{(interest)} \(\blc{\Vx[t]{}[]\delta_t\dd{t}}\)
is earned and added to the reserve.
\item \faIcon{plus-circle} \emph{(premium less expense)}
\((\gc{G_t}-\rc{e_t})\dd{t}\) is received and added to the reserve.
\item \faIcon{minus-circle} \emph{(\(\text{NAAR}\times \text{death prob.}\))}
\(\vc{\underbrace{(S_t+E_t-\Vx[t]{}[])}_{\text{NAAR}}\underbrace{\mu_{x+t}\dd{t}}_{\text{death
prob.}}}\) is subtracted from the reserve.
\end{itemize}
\item ``Summing'' up all these ``infinitesimal'' changes and adding it to the
starting reserve (\(\Vx[0]{}[]\)) gives the ending (time \(s\)) reserve:
\(\Vx[s]{}[]\).
\end{enumerate}
\begin{note}
\emph{(If you are interested)} See \textcite{dickson2019actuarial} for the
mathematical derivation of Thiele's differential equation.
\end{note}

\item \label{it:euler-method} While Thiele's differential equation is certainly of theoretical
interest, it is difficult to be directly used in practice, due to the high
difficulty of solving this differential equation. In view of this, often
\emph{numerical} methods for solving differential equation are utilized to
allow practical usages of Thiele's differential equation. One notable numerical
method is \defn{Euler's method}.

\emph{Key Formulas:} (\(h\): \defn{step size})
\begin{itemize}
\item \emph{\defn{Forward approximation}:} Change
\(\displaystyle \dv{}{t}\Vx[t]{}[]\to\frac{\Vx[t+h]{}[]-\Vx[t]{}[]}{h}\). This gives
\[
\frac{\Vx[t+h]{}[]-\Vx[t]{}[]}{h}
\approx \Vx[t]{}[]\delta_t+(G_t-e_t)
-(S_t+E_t-\Vx[t]{}[])\mu_{x+t}.
\]
\item \emph{\defn{Backward approximation}:} Change \(\displaystyle
\dv{}{t}\Vx[t]{}[]\to\frac{\Vx[t+h]{}[]-\Vx[t]{}[]}{h}\), \underline{and change \(t\to
t+h\) on the RHS}. This gives
\[
\frac{\Vx[t+h]{}[]-\Vx[t]{}[]}{h}
\approx \Vx[t+h]{}[]\delta_{t+h}+(G_{t+h}-e_{t+h})
-(S_{t+h}+E_{t+h}-\Vx[t+h]{}[])\mu_{x+t+h}.
\]
\end{itemize}
These formulas allow us to perform recursions in a similar way as the
\(1/m\)thly case, but via the Thiele's differential equation.
\end{enumerate}
\subsection{Asset Shares}
\label{subsect:asset-shares}
\begin{enumerate}
\item \emph{Definition:} The time-\(t\) \defn{asset share} is the share
\faIcon{pizza-slice} available in the ``\underline{actual} pool of money
\faIcon{coins}'' (accumulated from the \underline{actual} net cash flows
received before time \(t\)) per \underline{actual} survivor (or per policy in
force\footnote{As we shall see in \Cref{subsect:practical-issues}, sometimes it
is possible for a surviving policyholder to withdraw from his policy
voluntarily, making his policy not in force (but he is still a ``survivor'').
But to avoid unnecessary complications here, we will just use the term
``survivor'', with the understanding that it means more generally ``surviving
policy'', or policy in force.}) \faIcon{at} time \(t\):

\begin{align*}
\as_{t}&=\frac{\pvt{0}{\lx{x}\times \text{\underline{actual} NCFs received before time \(t\)}}\times (1+i^{\mathrm{actual}})^{t}}
{\underbrace{\lx{x+t}^{\mathrm{actual}}}_{\text{\underline{actual} number of survivors \faIcon{at} time \(t\)}}}\\
&=\frac{\lx{x}\times \pvt{0}{\text{\underline{actual} NCFs received}}\times (1+i^{\mathrm{actual}})^{t}}{\lx{x+t}^{\mathrm{actual}}} \\
&=\frac{\pvt{0}{\text{\underline{actual} NCFs received}}}{(v^{\mathrm{actual}})^{t}\px[t]{x}^{\mathrm{actual}}}
\end{align*}
where:
\begin{itemize}
\item \(\px[t]{x}^{\mathrm{actual}}=\lx{x+t}^{\mathrm{actual}}/\lx{x}\) is the
``actual survival probability'', i.e., the \underline{actual} fraction of
survivors \faIcon{at} time \(t\), among the people initially aged \(x\). \begin{note}
Similar meanings apply for ``\(\qx[t]{x}^{\mathrm{actual}}\)''.
\end{note}
\item \(i^{\mathrm{actual}}\) is the \underline{actual} effective interest rate
per annum. When the actual annual interest rates are different for different
years, we can modify the accumulation factor \((1+i^{\mathrm{actual}})^{t}\)
accordingly.
\end{itemize}

\begin{remark}
\item We have the following convention about whether a cash flow precisely
\faIcon{at} time \(t\) should be included as cash flow ``before'' time \(t\):
\begin{itemize}
\item \underline{Benefits} and \underline{benefit}-related expenses (i.e.,
settlement expenses) \faIcon{at} time \(t\): included.
\item \underline{Premiums} and \underline{premium}-related expenses (i.e.,
initial and renewal expenses) \faIcon{at} time \(t\): NOT included.
\end{itemize}
\item Since actual quantities are to be used, the earliest time at which we can
\emph{possibly} compute time-\(t\) asset share is time \(t\).
\end{remark}

\item \emph{Key formulas:}
\begin{enumerate}
\item \emph{(zero initial asset share)} We always have \(\as_{0}=0\), whenever
there is no benefit at time \(0\) (always the case virtually).

\item \label{it:asset-share-recurs} \emph{(recursive formula)} When \(t\) is an
integer time,
\[
(\as_{t}+G_t-e_t^{\mathrm{actual}})(1+i^{\mathrm{actual}}_t)
=\qx{x+t}^{\mathrm{actual}}(S_{t+1}+E_{t+1}^{\mathrm{actual}})
+\px{x+t}^{\mathrm{actual}}\as_{t+1}
\]
where the quantities with ``\(\mathrm{actual}\)'' superscript are the actual
ones (e.g., \(i_t^{\mathrm{actual}}\) refers to the actual time-\(t\) annual
interest rate). It takes a very similar form as the usual policy value
recursion.
\end{enumerate}
\end{enumerate}
\subsection{Practical Issues}
\label{subsect:practical-issues}
\begin{enumerate}
\item \textbf{Policy alteration.} Some terminologies:
\begin{itemize}
\item A policy is \defn{surrendered} when some of the policy value is paid to
the withdrawing policyholder, and is \defn{lapsed} if no payment is made.
\item The amount paid upon policy surrender is called the \defn{surrender value}.
\item A \defn{paid-up policy} is a policy where premium payments stop early
(they are \underline{paid up}) but remains in force. Because of this, the
benefit is usually reduced, and the reduced value is called the \defn{paid-up
sum insured}.
\end{itemize}
\begin{note}
Unless otherwise specified, we shall assume that there are no policy
lapses/surrenders.
\end{note}

\item \label{it:value-policy-alt} \emph{Basic principle for valuation upon
policy alteration:}
\[
\text{EPV of \emph{original} future loss at alteration time}
=\text{EPV of \emph{revised} future loss at alteration time},
\]
or
\[
\boxed{\text{Policy value of \emph{original} policy at alteration time}
=\text{Policy value of \emph{revised} policy at alteration time}}.
\]
For example, if a policyholder requests to alter his term life insurance of
\num{1000000} to a whole life insurance, when he is aged 65 and the term life
insurance has 10 years policy term remaining, then the benefit amount \(W\) for
the whole life insurance would be given by
\[
\num{1000000}\Ax{\itop{65}:\angl{10}}=W\Ax{65}
\]
according to the basic principle.

\item In practice, there may be various costs and charges upon policy
alteration. Such costs and charges may be regarded as a loss incurred for the
revised policy, at the policy alteration time. In this example, if there is a
cost of 5\% of the original policy value incurred for policy alteration, then
the equation would become
\[
\num{1000000}\Ax{\itop{65}:\angl{10}}=W\Ax{65}+\underbrace{0.05\times
\qty(\num{1000000}\Ax{\itop{65}:\angl{10}})}_{\mathclap{\text{EPV of extra loss for revised policy}}},
\]
which results in a smaller benefit \(W\).

\item \textbf{Negative policy value.} Negative policy value often arises in
early time period, which is typically caused by a relatively large initial
expense. To illustrate this, suppose the conditions for the equality of
retrospective and usual policy values are satisfied. Then, consider
\(\Vx[t]{}[]\) for some very small \(t\), say \(t=0.0001\). By the
retrospective approach,
\begin{align*}
\Vx[t]{}[]&=\frac{\epvt{0}{\text{past gross
premiums}}-\overbrace{\epvt{0}{\text{past benefits and
expenses}}}^{\text{involving initial expense}}}{\Ex[t]{x}} \\
&\approx -\text{initial expense}<0\qquad\text{(very small \(t\), very large initial expense)}.
\end{align*}
It is not prudent to assign positive value to a policy having negative policy
value, despite the apparent expected future gain, because the policyholder can
\textbf{lapse} the policy before such gain is captured! Hence, prudent
actuaries would suggest to just assign a \emph{zero} value to such policy.
\end{enumerate}

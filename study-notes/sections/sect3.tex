\section{Multiple Decrement Models}
\label{sect:mult-decr-models}
\begin{enumerate}
\item Multiple decrement model is about generalization by classifying different
ways of death/decrement. This generalization can be helpful as some policies
may provide different benefit amounts for deaths due to different causes. The
good \faIcon[regular]{thumbs-up} thing is that multiple decrement model can
actually be seen as a special case of multiple state model, so results in
\Cref{sect:mult-state-models} can be recycled. Hence, we will concentrate on
some concepts and formulas specialized for the multiple decrement model here.

\item A multiple decrement model can be graphically represented as follows.
\begin{center}
\begin{tikzpicture}
\node[draw, minimum width=3cm, minimum height=1.5cm] () at (1.5,0.5) {Active (0)};
\node[draw, minimum width=3cm, minimum height=1.5cm] () at (6.5,4.5) {Decrement 1 (1)};
\node[draw, minimum width=3cm, minimum height=1.5cm] () at (6.5,2.5) {Decrement 2 (2)};
\node[draw, minimum width=3cm, minimum height=1.5cm] () at (6.5,-1.5) {Decrement \(m\) (\(m\))};
\draw[-Latex] (3,0.5) -- (5,4.5);
\draw[-Latex] (3,0.5) -- (5,2.5);
\draw[-Latex] (3,0.5) -- (5,-1.5);
\node[] () at (4,0.5) {\(\vdots\)};
\node[] () at (6.5,0.5) {\(\vdots\)};
\end{tikzpicture}
\end{center}
\begin{note}
``Active'' means that the life is still ``active'' in the policy.
\end{note}

For obvious reasons, all the arrows here are one-directional. Note also that
there can only be \emph{at most one} change in states for a multiple decrement
model, namely transition from state \(0\) (active) to one of the decrement
states.  When the life transits from state \(0\) to state \(j\ne 0\) (decrement
\(j\)), we say that \defn{decrement \(j\) occurs}.  Throughout
\Cref{sect:mult-decr-models}, we will work in \emph{continuous-time} multiple
decrement models (i.e., decrement can occur at any time), unless otherwise
specified.
\end{enumerate}
\subsection{Probabilistic Calculations}
\label{subsect:mult-decr-prob-calc}
\begin{enumerate}
\item As multiple decrement model aims to classify different ways of ``death'',
terminologies and concepts exclusively related to death and survival from
STAT3901 would make sense in multiple decrement models generally, unlike the
multiple state model case. Therefore, we can now generalize concepts related to
life table in STAT3901, such as those ``\(\lx{}\)''s and ``\(\dx{}\)''s, and
fractional age assumptions.

\item \textbf{Multiple decrement tables.} Naturally, the first thing to be
generalized is the concept of \emph{life table} itself. Here, instead of
considering ``life table'', we will consider \emph{multiple decrement
table}\footnote{As suggested by the name of this model, we care more about
decrements than ``life'', so we use this terminology.}, which takes the
following form:
\begin{center}
\begin{tabular}{ccccc}
\toprule
Age \(x\)&\(\lx{x}[00]\)&\(\dx{x}[01]\)&\(\dx{x}[02]\)&\(\dx{x}[0\bullet]\) \\
\midrule
30&10000&50.25&40&\fpeval{50.25+40} \\
31&\fpeval{10000-50.25-40}&60&35.5&\fpeval{60+35.5} \\
32&\fpeval{10000-50.25-40-60-35.5}&70&40&\fpeval{70+40} \\
33&\fpeval{10000-50.25-40-60-35.5-70-40}&100&45&\fpeval{100+45} \\
\vdots&\vdots&\vdots&\vdots&\vdots \\
\bottomrule
\end{tabular}
\end{center}
The meanings of the notations are:
\begin{itemize}
\item \(\lx{x}[00]\): expected number of active lives \faIcon{at} age \(x\).
\item \(\dx{x}[0j]\): expected number of decrements \(j\) between ages \(x\) and \(x+1\).
\item \(\dx{x}[0\bullet]\): expected number of decrements between ages \(x\)
and \(x+1\), i.e., sum of all \(\dx{x}[0j]\)'s. Note that
\(\dx{x}[0\bullet]=\lx{x}[00]-\lx{x+1}[00]\).
\end{itemize}

\item \label{it:mult-decr-table-prob-fmlas} \textbf{Formulas of probabilistic
quantities based on multiple decrement table.} Like life table in STAT3901, the
primary purpose of multiple decrement table is to allow us to compute
probabilistic quantities based on the data given in the table. Here are some
key formulas (\(j\ne 0\) denotes a label for decrement state):
\begin{center}
\begin{tabular}{cccc}
\toprule
Quantity&using \(\lx{x}[00]\)&Using \(\dx{x}[0j]\)&Using \(\dx{x}[0\bullet]\) \\
\midrule
\(\px[t]{x}[00]=\px[t]{x}[\overline{00}]\)&\(\displaystyle \frac{\lx{x+t}[00]}{\lx{x}[00]}\)&\diagbox[dir=NE]{}{}&\diagbox[dir=NE]{}{}\\
\(\px[t]{x}[0j]\)&\diagbox[dir=NE]{}{}
&\(\displaystyle \frac{\overbrace{\dx{x}[0j]+\dotsb+\dx{x+t-1}[0j]}^{\text{\(t\) terms}}}{\lx{x}[00]}
\overset{(t=1)}{=}\frac{\dx{x}[0j]}{\lx{x}[00]}
\)
&\diagbox[dir=NE]{}{} \\
\(\px[t]{x}[0\bullet]\)&\(\displaystyle 1-\frac{\lx{x+t}[00]}{\lx{x}[00]}\)
&\diagbox[dir=NE]{}{}&\(\displaystyle \frac{\overbrace{\dx{x}[0\bullet]+\dotsb+\dx{x+t-1}[0\bullet]}^{\text{\(t\) terms}}}{\lx{x}[00]}
\overset{(t=1)}{=}\frac{\dx{x}[0\bullet]}{\lx{x}[00]}\) \\
\(\px[u|t]{x}[0j]\)&\diagbox[dir=NE]{}{}
&\(\displaystyle \frac{\overbrace{\dx{x+u}[0j]+\dotsb+\dx{x+u+t-1}[0j]}^{\text{\(t\) terms}}}{\lx{x}[00]}
\overset{(t=1)}{=}\frac{\dx{x+u}[0j]}{\lx{x}[00]}\)&\diagbox[dir=NE]{}{} \\
\(\px[u|t]{x}[0\bullet]\)&\(\displaystyle \frac{\lx{x+u}[00]-\lx{x+u+t}[00]}{\lx{x}[00]}\)
&\diagbox[dir=NE]{}{}&\(\displaystyle \frac{\overbrace{\dx{x+u}[0\bullet]+\dotsb+\dx{x+u+t-1}[0\bullet]}^{\text{\(t\) terms}}}{\lx{x}[00]}
\overset{(t=1)}{=}\frac{\dx{x+u}[0\bullet]}{\lx{x}[00]}\) \\
\midrule
\(\mu_x^{0j}\)&\(\displaystyle -\frac{\dv{}{x}\lx{x}[0j]}{\lx{x}[00]}\)
&\diagbox[dir=NE]{}{}&\diagbox[dir=NE]{}{} \\
\(\mu_x^{0\bullet}\)&\(\displaystyle -\frac{\dv{}{x}\lx{x}[00]}{\lx{x}[00]}\)
&\diagbox[dir=NE]{}{}&\diagbox[dir=NE]{}{} \\
\bottomrule
\end{tabular}
\end{center}
\begin{remark}
\item \(\lx{x}[0j]\) denotes the expected number of lives who are (i) active
\faIcon{at}  age \(x\) and (ii) \emph{ultimately} fall into decrement \(j\).
Note that \(\lx{x}[00]\) is the sum of all \(\lx{x}[0j]\)'s.
\item We always have \(\px[t]{x}[\overline{00}]=\px[t]{x}[00]\) in multiple
decrement model, because it is impossible to go back to state \(0\) once the
life leaves it.
\end{remark}

Here, \(\px[t]{x}[00]\), \(\px[t]{x}[0j]\), and \(\mu_x^{0j}\) still possess
the same meanings from \Cref{sect:mult-state-models}. The notation
\(\mu_x^{0\bullet}\) is the sum of all \(\mu_x^{0j}\)'s. \begin{note}
Sometimes the force of transition \(\mu_x^{0j}\) is called \defn{force of
decrement} here.
\end{note}

For the meanings of the rest of probability notations, see the table below.
\begin{center}
\begin{tabular}{ll}
\toprule
Notation with formulas&Probability that \((x)\) ...\\
\midrule
\(\px[t]{x}[0\bullet]=\sum_{j}^{}\px[t]{x}[0j]=1-\px[t]{x}[00]\)&\rc{has any decrement} within \(t\) years \\
\midrule
\(\px[u|t]{x}[0j]=\gc{\px[u]{x}[00]}\times \rc{\px[t]{x+u}[0j]}
=\rc{\px[u+t]{x}[0j]}-\rc{\px[u]{x}[0j]}
\)&
\makecell[l]{
\gc{stays active} for \(u\) years and \\
\rc{has decrement \(j\)} in the subsequent \(t\) years
} \\
\midrule
\(\px[u|t]{x}[0\bullet]
=\gc{\px[u]{x}[00]}\times \rc{\px[t]{x+u}[0\bullet]}
=\rc{\px[u+t]{x}[0\bullet]}-\rc{\px[u]{x}[0\bullet]}
=\gc{\px[u]{x}[00]}-\gc{\px[u+t]{x}[00]}
\)&
\makecell[l]{
\gc{stays active} for \(u\) years and \\
\rc{has any decrement} in the subsequent \(t\) years
}\\
\bottomrule
\end{tabular}
\end{center}
\item \textbf{Specialized transition probability formula.} The general
transition probability formula in \labelcref{it:gen-tran-prob-fmla} can be
simplified substantially in the multiple decrement model setting. First recall
the general formula:
\[
\px[t]{x}[ij]=\int_{0}^{t}\qty(\sum_{k\ne j}^{}\px[s]{x}[ik]\mu_{x+s}^{kj})
\px[t-s]{x+s}[\overline{jj}]\dd{s}.
\]
Certainly, when \(i\ne 0\), the transition probability \(\px[t]{x}[ij]\)
always \(1\) if \(j=i\) and always \(0\) otherwise (we do not even need to use the
general formula above to tell this!). This is not too interesting, so we focus
on the case when \(i=0\).

Since \(\px[t]{x}[00]=\px[t]{x}[\overline{00}]\), we can just apply the
occupancy probability formula to compute \(\px[t]{x}[00]\). So henceforth we
focus on on the transition probability of the form \(\px[t]{x}[0j]\) where
\(j\ne 0\). Note that:
\begin{itemize}
\item \(\mu_{x+s}^{kj}=0\) whenever \(k\ne 0\), since it is impossible to
transit from one decrement state to another decrement state.
\item \(\px[t-s]{x+s}[\overline{jj}]=1\) because it is impossible to leave a
decrement state.
\end{itemize}
Therefore, we can simplify the general formula as:
\[
\boxed{\px[t]{x}[0j]=\int_{0}^{t}\px[s]{x}[00]\mu_{x+s}^{0j}\dd{s}}.
\]
\item \textbf{Fractional age assumptions.} Like life table, multiple decrement
table often contains values for \emph{integer} ages only. Hence, to compute
probabilistic quantities involving fractional ages and years based on multiple
decrement table, a \emph{fractional age assumption} is needed. Here we will
discuss two common ones\footnote{Let's forget about the somewhat ``unpleasant''
and seldom used \emph{Balducci
assumption}...\faIcon[regular]{grin-beam-sweat}}:
\begin{itemize}
\item \defn{uniform distribution of decrements} (UDD again): \\
\(\px[t]{x}[0j]=t\px{x}[0j]\) for \underline{any decrement \(j\)} and any \(0\le t\le 1\).
\item \defn{constant forces of decrement} (CF): \\
\(\mu_{x+t}^{0j}=\mu_{x}^{0j}\) for \underline{any decrement \(j\)} and any
\(0\le t<1\).
\end{itemize}

\item \label{it:faa-fmlas} \textbf{Key formulas for fractional age
assumptions.} Below, we assume that \(x\) is an integer age and \(t\) is any
value in \([0,1]\).
\begin{center}
\begin{tabular}{lcc}
\toprule
&UDD&CF\\
\midrule
\(\px[t]{x}[00]\)&\diagbox[dir=NE]{}{}
&\(=\px[t]{x+s}[00]=(\px[]{x}[00])^{t}\)
\\
\midrule
\(\px[t]{x}[0j]\)&\(=\px[u|t]{x}[0j]=t\px{x}[0j]\)
&\(\displaystyle \underset{\text{(``share'' of \(j\))}}
{\frac{\px{x}[0j]}{\px{x}[0\bullet]}}\times \px[t]{x}[0\bullet]\)\\
\midrule
\(\px[t]{x}[0\bullet]\)&\(=\px[u|t]{x}[0\bullet]=t\px{x}[0\bullet]\)
&\diagbox[dir=NE]{}{}\\
\midrule
\(\mu_{x+t}^{0j}\)&\(\displaystyle \frac{\px[]{x}[0j]}{1-t\px[]{x}[0\bullet]}\)
&\(\displaystyle 
\underset{\text{(``share'' of \(j\))}}
{\frac{\px{x+t}[0j]}{\px{x+t}[0\bullet]}}\times\mu_{x+t}^{0\bullet}\)\\
\bottomrule
\end{tabular}
\end{center}
For finding quantities whose formulas are not found in the table, a general
strategy is to express them in terms of some quantities with known formulas.
For example:
\begin{itemize}
\item UDD: \(\px[t]{x}[00]=1-\px[t]{x}[0\bullet]=1-t\cdot \px{x}[0\bullet]\).
\item UDD: (\(0\le s+t\le 1\)) \(\displaystyle \px[t]{x+s}[0j]=\frac{\px[s|t]{x}[0j]}{\px[s]{x}[00]}
=\frac{t\px{x}[0j]}{1-s\px{x}[0\bullet]}\).
\item CF: \(\px[t]{x}[0\bullet]=1-\px[t]{x}[00]=1-(\px{x}[00])^{t}\).
\end{itemize}
\item \textbf{Random variable approach to multiple decrement model.} So far we
have dealt with multiple decrement model as a special case of multiple state
model. But this is not the only way of handling multiple decrement model.
Another notable approach is to introduce suitable \emph{random variables} for a
life aged \(x\) that is currently active:
\begin{itemize}
\item \(T=T_x\): time of occurrence of decrement.
\item \(J=J_x\): label of the decrement occurred (or cause of the eventual
decrement). For a multiple decrement model with \(m\) decrements, it is a
discrete random variable taking value in \(\{1,\dotsc,m\}\).
\end{itemize}

\item \label{it:isolate-decr} We can define \(T\) more precisely as follows.
First, for any \(j=1,\dotsc,m\) (assuming there are \(m\) decrements) let
\(T_j\), denote the ``notional'' time of occurrence of decrement \(j\)
\underline{in isolation}, i.e., the time of transition from state \(0\) to
state \(j\) after ``extracting'' these two states from the multiple decrement
model (without changing the probabilistic behaviour of transition from state
\(0\) to state \(j\)):
\begin{center}
\begin{tikzpicture}
\node[draw, minimum width=3cm, minimum height=1.5cm] () at (1.5,0.5) {Active (0)};
\node[draw, minimum width=3cm, minimum height=1.5cm] () at (6.5,2.5) {Decrement \(j\) (\(j\))};
\draw[-Latex] (3,0.5) -- (5,2.5);
\node[] () at (5.5,0.5) {Time of transition: \(T_j\)};
\end{tikzpicture}
\end{center}
Then, the random variable \(T\) is defined by \(T=\min\{T_1,\dotsc,T_m\}\). In
other words, the observed decrement in the multiple decrement model is seen as
the decrement with the shortest ``notional'' time of occurrence in isolation,
and such time in isolation is taken as the occurrence time in the multiple
decrement model. One (reasonable) assumption to be imposed here is that
\(T_1,\dotsc,T_m\) are independent.

By default, the multiple decrement model is continuous-time, thus
\(T_1,\dotsc,T_m\) are all continuous. In \labelcref{it:discrete-decr}, we
will deal with a case where some of \(T_1,\dotsc,T_m\) are \emph{discrete},
which is trickier \warn{}.

\item \label{it:rv-mult-decr-prob-fmlas} \textbf{Formulas for probabilistic
quantities based on random variable approach.}
In the following table, we list some key formulas for probabilistic quantities
under this random variable approach. (Here we assume there are \(m\) decrements.)

\begin{center}
\begin{tabular}{cc}
\toprule
Quantity&Formula \\
\midrule
marginal density function \(f_T(t)\)&\(\displaystyle \sum_{j=1}^{m}f_{T,J}(t,j)\) \\
marginal mass function \(f_J(j)\)&\(\displaystyle \int_{0}^{\infty}f_{T,J}(t,j)\dd{t}=\lim_{t\to \infty}\px[t]{x}[0j]\) \\
\(\px[t]{x}[00]=\px[t]{x}[\overline{00}]\)&\(\displaystyle \prob{T>t}=\int_{t}^{\infty}f_{T}(s)\dd{s}\) \\
\(\px[t]{x}[0j]\)&\(\displaystyle \prob{T\le t\cap J=j}=\int_{0}^{t}f_{T,J}(s,j)\dd{s}\) \\
\(\px[t]{x}[0\bullet]\)&\(\displaystyle \prob{T\le t}=\int_{0}^{t}f_{T}(s)\dd{s}\) \\
\(\px[u|t]{x}[0j]\)&\(\displaystyle \prob{u<T\le u+t\cap J=j}=\int_{u}^{u+t}f_{T,J}(s,j)\dd{s}\) \\
\(\px[u|t]{x}[0\bullet]\)&\(\displaystyle \prob{u<T\le u+t}=\int_{u}^{u+t}f_{T}(s)\dd{s}\) \\
\(\mu_{x+t}^{0j}\)&\(\displaystyle \frac{f_{T,J}(t,j)}{\prob{T>t}}=\frac{f_{T,J}(t,j)}{\px[t]{x}[00]}\) \\
\(\mu_{x+t}^{0\bullet}\)&\(\displaystyle \frac{f_{T}(t)}{\prob{T>t}}=\frac{f_{T}(t)}{\px[t]{x}[00]}
=-\frac{\dv{}{t}\px[t]{x}[00]}{\px[t]{x}[00]}\) \\
\(\eringx{x}[00]=\expv{T}\)&\(\displaystyle \int_{0}^{\infty}\px[t]{x}[00]\dd{t}\) \\
\bottomrule
\end{tabular}
\end{center}
\begin{note}
Here \(f_{T,J}\) denotes the joint probability function of \(T\) and \(J\).
\end{note}

One can also define the ``curtate'' time of occurrence of decrement
\(K\triangleq\lfloor T\rfloor\) and develop the corresponding formulas. But
things are starting to feel overwhelming, let us stop here for
now...\faIcon[regular]{grin-beam-sweat} (See \labelcref{it:double-indemn} for
an example usage of \(K\).)

\item \textbf{Associated single decrement model.} In
\labelcref{it:isolate-decr}, we have ``extracted'' models containing only
single decrement from a multiple decrement model. Each such single decrement
model obtained in this way is called an \defn{associated single decrement
model}. One important property is that every associated single decrement model
preserves the probabilistic behaviour of transition from the active state (state
\(0\)) to the decrement state in the multiple decrement model. More
specifically, we require the force of decrement \(\mu_{x}^{0j}\) to be always the
same for both the multiple decrement model and the associated single decrement
model (for decrement \(j\)).

Transition probabilities in the associated single decrement model for decrement
\(j\) are defined based on the random variable \(T_j\) (time of occurrence of
decrement \(j\) in the associated single decrement model) and have special
notations:
\begin{itemize}
\item \emph{Transition probability from state 0 to state 0:} \(\px[t]{x}[\prime 00(j)]=\prob{T_j>t}\).
\item \emph{Transition probability from state 0 to state \(j\):} \(\px[t]{x}[\prime 0j]=\prob{T_j\le t}\).
\end{itemize}
\begin{note}
The decrement probability \(\px[t]{x}[\prime 0j]\) is sometimes called
\defn{independent probability}, and the decrement probability \(\px[t]{x}[0j]\)
is sometimes called \defn{dependent probability}. To better understand these
terminologies, consider:
\[
\px[t]{x}[0j]=\int_{0}^{t}\underbrace{\px[t]{x}[00]}_{\px[t]{x}[\overline{00}]}\mu_{x+s}^{0j}\dd{s}
=\int_{0}^{t}\exp\qty(-\int_{0}^{s}\mu_{x+r}^{0\bullet}\dd{r})\mu_{x+s}^{0j}\dd{s}.
\]
From this expression, we can see that \(\px[t]{x}[0j]\) is
\underline{dependent} on the forces of transition for other decrements, not
just decrement \(j\). On the other hand, \(\px[t]{x}[\prime 0j]\) is
\underline{independent} from the forces of transition for decrements other than
decrement \(j\).
\end{note}

\item \label{it:mult-decr-asso-single-decr-relate} \textbf{Relationships between
multiple decrement model and associated single decrement model.} We can relate
the transition probability in the multiple decrement model with the ones here
via the following equation (assuming there are \(m\) decrements):
\[
\boxed{\px[t]{x}[00]=\prod_{j=1}^{m}\px[t]{x}[\prime 00(j)]}.
\]
\begin{pf}
Note that
\[
\px[t]{x}[00]=\prob{T>t}
=\prob{\min\{T_1,\dotsc,T_m\}>t}
=\prob{T_1>t\cap\dotsb\cap T_m>t}
=\prod_{j=1}^{m}\prob{T_j>t}
=\prod_{j=1}^{m}\px[t]{x}[\prime 00(j)].
\]
\end{pf}

One consequence of this relationship is that \(\px[t]{x}[0j]<\px[t]{x}[\prime
0j]\) when there are more than one decrement in effect for the multiple
decrement model (i.e., there is a positive force of decrement other than
\(\mu_{x}^{0j}\)). Intuitively, this happens due to the presence of
\emph{competing} decrements in multiple decrement model: If a life has a
decrement \(k\ne j\), then he cannot have the decrement \(j\) anymore. So in
this sense, decrements \(j\) and \(k\) are ``competing''.

\begin{pf}
When there are more than one decrement in effect, \(\px[t]{x}[\prime 00(k)]<1\)
for some \(k\ne j\). Hence, \(\vc{\px[t]{x}[00]}<\vc{\px[t]{x}[\prime 00(j)]}\). Thus,
\[
\px[t]{x}[0j]=\int_{0}^{t}\vc{\px[s]{x}[00]}\mu_{x+s}^{0j}\dd{s}
<\int_{0}^{t}\vc{\px[s]{x}[\prime 00(j)]}\mu_{x+s}^{0j}\dd{s}
=\px[t]{x}[\prime 0j].
\]
\end{pf}
\item \label{it:mult-decr-asso-single-decr-udd-relate} \textbf{Further relationships under UDD.} Without further assumptions, it
is not possible to directly obtain \(\px[t]{x}[0j]\) from \(\px[t]{x}[\prime
0j]\), and vice versa. So here, we will impose a further assumption, namely
UDD, which yields some nice formulas that allow us to go from one type of
quantity to another. Depending on whether we assume UDD in the multiple
decrement model (known as \defn{MUDD}) or in all the associated single
decrement models (known as \defn{SUDD}), the formulas would be different.

MUDD is just the usual UDD assumption discussed previously (namely assuming
\(\px[t]{x}[0j]=t\px{x}[0j]\) for every decrement \(j\) and any \(0\le t\le 1\)).
SUDD means imposing the UDD assumption on every associated single decrement
model, i.e., assuming \(\px[t]{x}[\prime 0j]=t\px{x}[\prime 0j]\) for every
decrement \(j\) and any \(0\le t\le 1\).

Formulas for relating \(\px{x}[0j]\) and \(\px{x}[\prime 0j]\) under MUDD
and SUDD are as follows (\(0\le t\le 1\)).

\begin{center}
\begin{tabular}{ccc}
\toprule
&MUDD&SUDD \\
\midrule
\(\px{x}[0j]\to\px{x}[\prime 0j]\)&
\(\displaystyle \px[t]{x}[\prime 0j]=1-(\px[t]{x}[00])^{\overset{\mathclap{\text{\footnotesize{(``share'' of \(j\))}}}}
{\px{x}[0j]/\px{x}[0\bullet]}}\)&(solve system of equations from below numerically) \\
\(\px{x}[\prime 0j]\to\px{x}[0j]\)&(solve the equation above)
&\(\displaystyle \px[t]{x}[0j]=\px{x}[\prime 0j]\qty(
t-\frac{t^{2}}{2}\sum_{k\ne j}^{}\px{x}[\prime 0k]
+\frac{t^{3}}{3}\sum_{k,\ell\ne j}^{}\px{x}[\prime 0k]\px{x}[\prime 0\ell]-\dotsb
)\) \\
\bottomrule
\end{tabular}
\end{center}
Often we are using these formulas with \(t=1\), but not always.

\item \textbf{More about formulas under SUDD.} Using the formulas under SUDD is
a bit tricky in practice. First, let us consider the formula for
\(\px{x}[\prime 0j]\overset{\text{SUDD}}{\to}\px{x}[0j]\). It would become more
complex (rather quickly!) as there are more decrements. In practice, this
formula is usually used when there are 2 or 3 decrements, so let us write down
the formulas more explicitly in these cases:
\begin{itemize}
\item \emph{2 decrements (1, 2):}
\[
\begin{cases}
\displaystyle \px[t]{x}[01]=\px{x}[\prime 01]\qty(t-\frac{t^{2}}{2}\px{x}[\prime 02])
\overset{(t=1)}{=}\px{x}[\prime 01]\qty(1-\frac{1}{2}\px{x}[\prime 02]), \\
\displaystyle \px[t]{x}[02]=\px{x}[\prime 02]\qty(t-\frac{t^{2}}{2}\px{x}[\prime 01])
\overset{(t=1)}{=}\px{x}[\prime 02]\qty(1-\frac{1}{2}\px{x}[\prime 01]).
\end{cases}
\]
\item \emph{3 decrements (1, 2, 3):}
\[
\begin{cases}
\displaystyle \px[t]{x}[01]=\px{x}[\prime 01]\qty(t-\frac{t^{2}}{2}\qty(\px{x}[\prime 02]+\px{x}[\prime 03])
+\frac{t^3}{3}\qty(\px{x}[\prime 02]\px{x}[\prime 03]))
\overset{(t=1)}{=}\px{x}[\prime 01]\qty(1-\frac{1}{2}\qty(\px{x}[\prime 02]+\px{x}[\prime 03])
+\frac{1}{3}\qty(\px{x}[\prime 02]\px{x}[\prime 03])), \\

\displaystyle \px[t]{x}[02]=\px{x}[\prime 02]\qty(t-\frac{t^{2}}{2}\qty(\px{x}[\prime 01]+\px{x}[\prime 03])
+\frac{t^3}{3}\qty(\px{x}[\prime 01]\px{x}[\prime 03]))
\overset{(t=1)}{=}\px{x}[\prime 02]\qty(1-\frac{1}{2}\qty(\px{x}[\prime 01]+\px{x}[\prime 03])
+\frac{1}{3}\qty(\px{x}[\prime 01]\px{x}[\prime 03])), \\

\displaystyle \px[t]{x}[03]=\px{x}[\prime 03]\qty(t-\frac{t^{2}}{2}\qty(\px{x}[\prime 01]+\px{x}[\prime 02])
+\frac{t^3}{3}\qty(\px{x}[\prime 01]\px{x}[\prime 02]))
\overset{(t=1)}{=}\px{x}[\prime 03]\qty(1-\frac{1}{2}\qty(\px{x}[\prime 01]+\px{x}[\prime 02])
+\frac{1}{3}\qty(\px{x}[\prime 01]\px{x}[\prime 02])).
\end{cases}
\]
\end{itemize}

Next, regarding the ``formula'' for
\(\px{x}[0j]\overset{\text{SUDD}}{\to}\px{x}[\prime 0j]\), let us explain more
about what is meant by ``numerically solving the system of equations''. Let us
use the 2 decrements case with \(t=1\) as an example (almost always the case when we are to
apply this method in practice).

From above, we have a system of equations for this case:
\[
\begin{cases}
\displaystyle \px{x}[01]=\px{x}[\prime 01]\qty(1-\frac{1}{2}\px{x}[\prime 02]), \\
\displaystyle \px{x}[02]=\px{x}[\prime 02]\qty(1-\frac{1}{2}\px{x}[\prime 01]),
\end{cases}
\]
which can be rearranged as
\[
\begin{cases}
\displaystyle \px{x}[\prime 01]=\frac{\px{x}[01]}{1-(1/2)\px{x}[\prime 02]}, \\
\displaystyle \px{x}[\prime 02]=\frac{\px{x}[02]}{1-(1/2)\px{x}[\prime 01]}.
\end{cases}
\]
It is difficult to solve this system analytically in general, so we would apply
the following \emph{iterative} process to solve it \emph{numerically}:
\begin{enumerate}[label={(\arabic*)}]
\item \emph{(Initialization)} Assign \(\px[][(0)]{x}[\prime 01]\leftarrow\px{x}[01]\) and
\(\px[][(0)]{x}[\prime 02]\leftarrow \px{x}[02]\) initially.
\item \emph{(Iteration)} For \(i=1,\dotsc,N\) (\(N\): number of iterations):
\[
\begin{cases}
\displaystyle \px[][(i)]{x}[\prime 01]\leftarrow\frac{\px{x}[01]}{1-(1/2)(\px[][(i-1)]{x}[\prime 02])}, \\
\displaystyle \px[][(i)]{x}[\prime 02]\leftarrow\frac{\px{x}[02]}{1-(1/2)(\px[][(i-1)]{x}[\prime 01])}.
\end{cases}
\]
\item \emph{(Return final results)} The numerical solution is given by \(\px[][(N)]{x}[\prime 01]\)
and \(\px[][(N)]{x}[\prime 02]\).
\end{enumerate}
\begin{note}
Instead of fixing a specific number of iterations \(N\) to be performed, one
can also iterate until a certain stopping criterion is met (e.g., the assigned
values of \(\px{x}[\prime 01]\) and \(\px{x}[\prime 02]\) remain unchanged up
to a specific number of decimal places after an iteration).
\end{note}
\item \label{it:discrete-decr} \textbf{Presence of discrete decrements.}
So far we have exclusively discussed multiple decrement model in the
\emph{continuous-time} setting. Now, in the last part of
\Cref{subsect:mult-decr-prob-calc}, we will discuss calculations outside this
setting. More specifically, we consider the case where some decrements are
discrete, i.e., some of \(T_1,\dotsc,T_m\) (as defined in
\labelcref{it:isolate-decr}) are discrete. Here we will introduce formulas for
computing dependent decrement probability \(\px[t]{x}[0j]\) in terms of
independent decrement probabilities:
\begin{itemize}
\item \emph{Decrement \(j\) is continuous (i.e., \(T_j\) is continuous):}
\[
\px[t]{x}[0j]=\int_{0}^{t}\px[s]{x}[00]\mu_{x+s}^{0j}\dd{s}
=\boxed{\int_{0}^{t}\qty(\prod_{k=1}^{m}\px[s]{x}[\prime 00(k)])\mu_{x+s}^{0j}\dd{s}}.
\]
\begin{itemize}
\item \emph{Special case:} If we further assume SUDD for decrement \(j\) and
\(0\le t\le 1\), then
\[
\boxed{\px[t]{x}[0j]=\px{x}[\prime 0j]\int_{0}^{t}\prod_{k\ne j}^{}\px[s]{x}[\prime 00(j)]\dd{s}}
\]
since \(\px[s]{x}[\prime 00(j)]\mu_{x+s}^{0j}=\px{x}[\prime 0j]\) in this case (by \labelcref{it:faa-fmlas}).
\end{itemize}
\item \emph{(Tricky \warn{}) Decrement \(j\) is discrete (i.e., \(T_j\) is discrete):}
\begin{enumerate}[label={(\arabic*)}]
\item For each possible time of decrement \(s\) (e.g., \(t=1,2,,\dotsc\) if
decrement \(j\) can only occur at the end of each year), the \textbf{dependent}
probability of having decrement \(j\) \faIcon{at} time \(s\) is
\begin{align*}
\prob{T=s\cap J=j}&=\prob{\qty(\bigcap_{k\ne j}T_k>s)\cap T_j=s} \\
&=\qty(\prod_{k\ne j}^{}\prob{T_k>s})\prob{T_j=s} \\
&=\boxed{\qty(\prod_{k\rc{\ne j}}^{}\px[s]{x}[\prime 00(k)])
\times \text{\textbf{independent} probability of having decrement \(j\) \faIcon{at} time \(s\)}}. \\
\end{align*}
Intuitively, the first term \(\prod_{k\ne j}^{}\px[t]{x}[\prime 00(k)]\) is the
probability of ``surviving through'' all decrements other than decrement \(j\),
so the idea is that the life needs to \underline{both} ``survive through''
all other decrements for \(s\) years, \underline{and} has decrement \(j\)
\faIcon{at} time \(s\).
\item We then have
\[
\px[t]{x}[0j]=\prob{T\le t\cap J=j}=\boxed{\sum_{s\le t}^{}\prob{T=s\cap J=j}}.
\]
\end{enumerate}
\end{itemize}
\end{enumerate}
\subsection{Insurance and Annuity EPV Calculations}
\begin{enumerate}
\item While we have spent quite a lot of time to discuss probabilistic
calculations in multiple decrement model, we will go through the remaining
parts of \Cref{sect:mult-decr-models} in a much shorter time (fortunately!), as
there are not much ``new'' things to be discussed: Most results are just
specialized versions of what have seen in \Cref{sect:mult-state-models}.

\item \textbf{EPV formulas.} EPV formulas in multiple decrement model are
essentially special cases of the ones in \labelcref{it:mult-state-epv-fmlas},
and can again be developed based on the famous general EPV calculation
formula. So we will just list some more frequently used EPV formulas below for
reference (assuming amount of each benefit payment/benefit rate is \(1\)):
\begin{center}
\begin{tabular}{lll}
\toprule
Type&Discrete&Continuous \\
\midrule
Permanent insurance&
\(\displaystyle \Ax{x}[0k]=\sum_{t=0}^{\infty}\vc{v^{t+1}}\mgc{\px[t]{x}[00]\px{x+t}[0k]}\)
&\(\displaystyle \Ax*{x}[0k]=\int_{0}^{\infty}\vc{e^{-\delta t}}\mgc{\px[t]{x}[00]\mu_{x+t}^{0k}\dd{t}}\) \\
Permanent annuity&
(due) \(\displaystyle \ax**{x}[00]=\sum_{t=0}^{\infty}\vc{v^{t}}\mgc{\px[t]{x}[00]} \)
& \(\displaystyle \ax*{x}[00]=\int_{0}^{\infty}\vc{e^{-\delta t}}\mgc{\px[t]{x}[00]}\brc{\dd{t}} \)
\\
\(n\)-year term insurance&
\(\displaystyle \Ax{\endowxn}[0k]=\sum_{t=0}^{\blc{n-1}}\vc{v^{t+1}}\mgc{\px[t]{x}[00]\px{x+t}[0k]}\)
&\(\displaystyle \Ax*{\endowxn}[0k]=\int_{0}^{\blc{n}}\vc{e^{-\delta t}}\mgc{\px[t]{x}[00]\mu_{x+t}^{0k}\dd{t}}\) \\
\(n\)-year temporary annuity&
(due) \(\displaystyle \ax**{\endowxn}[00]=\sum_{t=0}^{\blc{n-1}}\vc{v^{t}}\mgc{\px[t]{x}[00]}\)
& \(\displaystyle \ax*{\endowxn}[00]=\int_{0}^{\blc{n}}\vc{e^{-\delta t}}\mgc{\px[t]{x}[00]}\brc{\dd{t}} \)
\\
\bottomrule
\end{tabular}
\end{center}
\end{enumerate}
\subsection{Premium and Policy Value Calculations}
\begin{enumerate}
\item \textbf{Loss random variable.} Here we are going to discuss the only
``new'' thing, namely \emph{loss random variable}. We have not discussed it in
the multiple state model setting, because dealing with it in such a general
setting is somewhat complicated and would not be too fruitful. But in the
multiple decrement model setting, the concept of loss random variable would
become more manageable and interesting.

A natural way to develop the loss \underline{random variable} in the multiple
decrement setting is to utilize the \underline{random variable} approach: The
time-\(t\) (present value of) \defn{future loss random variable} is defined by:
\begin{itemize}
\item \emph{time-\(t\) \defn{net future loss}:} \(L_t^n=\pvt{t}{\text{future
benefits}}-\pvt{t}{\text{future \textbf{net} premiums}}\)
\item \emph{time-\(t\) \defn{gross future loss}:} \(L_t^g=\pvt{t}{\text{future
benefits and expenses}}-\pvt{t}{\text{future \textbf{gross} premiums}}\)
\end{itemize}
Here we assume that the life (aged \(x\) \faIcon{at} time \(0\)) is active
\faIcon{at} time \(t\).

As one may expect, the terms would involve the time of occurrence of decrement
\(T=T_{x+t}\) and the cause-of-decrement random variable \(J=J_{x+t}\).
Indeed, the future loss often takes the following form (we use \(L_t^g\) as an
example):
\[
L_t^g=\begin{cases}
\underset{\text{(expression in terms of \(T\))}}{\pvt{t}{\text{future benefits and expenses \textbf{for decrement \(1\)}}}}
-\underset{\text{(expression in terms of \(T\))}}{\pvt{t}{\text{future gross premiums}}}
&\text{if \(J=1\)}, \\
\qquad\vdots& \\
\underset{\text{(expression in terms of \(T\))}}{\pvt{t}{\text{future benefits and expenses \textbf{for decrement \(m\)}}}}
-\underset{\text{(expression in terms of \(T\))}}{\pvt{t}{\text{future gross premiums}}}
&\text{if \(J=m\)},
\end{cases}
\]
where \(\pvt{t}{\text{future gross premiums}}\) is independent from \(J\) as
premiums are only payable at state \(0\) (active state).

\item \label{it:double-indemn} \textbf{Case study: whole life double indemnity
insurance.} To illustrate how we work with loss random variable in the multiple
decrement model, let us use a \emph{whole life double indemnity insurance} as
an example.

First we consider the fully continuous case, with the following terms:
\begin{itemize}
\item A benefit of \(2S\) is paid immediately at the occurrence of accidental death (decrement 1).
\item A benefit of \(S\) is paid immediately at the occurrence of non-accidental death (decrement 2).
\item Net premiums of rate \(P\) are paid continuously until (accidental or non-accidental) death.
\end{itemize}
Then, the time-\(t\) net future loss of this policy would be
\[
L_t^n=\begin{cases}
2Sv^{T-t}-P\ax*{\angl{T-t}}&\text{if \(J=1\)}, \\
Sv^{T-t}-P\ax*{\angl{T-t}}&\text{if \(J=2\)}.
\end{cases}
\]

On the other hand, for the fully discrete case with the following terms:
\begin{itemize}
\item A benefit of \(2S\) is paid at the end of year of accidental death (decrement 1).
\item A benefit of \(S\) is paid at the end of year of non-accidental death (decrement 2).
\item A net premium \(P\) is paid at the beginning of each year whenever the life insured is active.
\end{itemize}
Then, the time-\(t\) net future loss of this policy would be
\[
L_t^n=\begin{cases}
2Sv^{K+1-t}-P\ax**{\angl{K+1-t}}&\text{if \(J=1\)}, \\
Sv^{K+1-t}-P\ax**{\angl{K+1-t}}&\text{if \(J=2\)},
\end{cases}
\]
where \(t\) is an integer time and \(K\triangleq\lfloor T\rfloor\) is the
``curtate'' time of occurrence of decrement.

With the future loss random variables, we can then perform calculations about
their probabilistic quantities, premiums, and policy values, like what we did
in STAT3901. (You may want to review \faIcon{book-reader} the corresponding
parts in STAT3901, and try to ``translate'' the tasks there to the case here.)
Here the time-\(t\) policy value is given by:
\begin{itemize}
\item \emph{time-\(t\) net premium policy value:} \(\Vx[t]{}[n]=\expv{L_t^n}\).
\item \emph{time-\(t\) gross premium policy value:} \(\Vx[t]{}[g]=\expv{L_t^g}\).
\end{itemize}
The time-\(t\) policy value here coincides with the ``\(\Vx[t]{}[(0)]\)'' for
the multiple state model.
\end{enumerate}


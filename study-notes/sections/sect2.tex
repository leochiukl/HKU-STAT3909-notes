\section{Multiple State Models}
\label{sect:mult-state-models}
\begin{enumerate}
\item If \Cref{sect:policy-values} is described as the ``appetizer'', then what
we are going to cover in
\Cref{sect:mult-state-models,sect:mult-decr-models,sect:mult-life-models} would
be the ``main course'', with \Cref{sect:profit-analysis} being the ``dessert''.
\Cref{sect:mult-state-models,sect:mult-decr-models,sect:mult-life-models}
is essentially about \emph{generalizing} what we learnt in STAT3901 in several
ways:
\begin{itemize}
\item \emph{\Cref{sect:mult-state-models} (multiple state models):}
Generalization by classifying people into \underline{multiple states}.
\item \emph{\Cref{sect:mult-decr-models} (multiple decrement models):}
Generalization by classifying people \emph{dead} into \underline{multiple} ways
of death (or more properly, \underline{decrement}\footnote{Analogous to the
remark for asset share in \Cref{subsect:asset-shares}, it is possible for a
policyholder to exit from the policy without actually dying, e.g., through
lapsing the policy. So the term ``decrement'' (which refers generally to any
kind of ``quitting the policy'') is used instead of ``death'' here.}).
\item \emph{\Cref{sect:mult-life-models} (multiple life models):}
Generalization by incorporating \underline{multiple lives} in a single policy.
\end{itemize}
So, in some sense, we are repeating the STAT3901 course for \emph{three times},
one for each kind of the models here!\footnote{Perhaps this is a bit
exaggerated, as we will only repeat part of (but still quite a lot of!) the
topics covered in STAT3901 for each type of the models here.} While we have
spent \emph{one whole semester} for going through the STAT3901 course, we will
only spend \emph{less than one} semester for going through these three sections
(``\(3\times \text{STAT3901 course}\)'') \faIcon[regular]{grin-beam-sweat}.
Thus, as you can expect, the pace in STAT3909 will be \emph{much quicker}
\faIcon{angle-double-right} than the one in STAT3901, which should be
manageable \emph{if} you are familiar with the STAT3901 content. (Hopefully this
is the case...  \faIcon[regular]{grin-beam-sweat}; \warn{} If not, it is better
for you to take a moment and review \faIcon{book-reader} what you have learnt
in STAT3901 first, before proceeding further.)

\Cref{sect:profit-analysis} is about a more practically useful topic, namely
analyzing \emph{profits}. The ultimate goal of selling insurance products is to
make profits (right?), so certainly this is a topic deserving attention. We put
it at the last as we are going to apply some concepts learnt in the prior
sections to the profit analysis. This section is perhaps (?) the easiest part
in STAT3909 and can be your ``grade saver'' \faIcon{cross}.\footnote{Note
however that \(\text{``easiest''}\ne\text{``easy''}\). Certainly you still need
to pay (substantial) effort on studying this section in order for it to
``save your grade'' (hopefully).}

\item To guide our discussions about the generalizations in
\Cref{sect:mult-state-models,sect:mult-decr-models,sect:mult-life-models}, we
will organize the contents covered in each of
\Cref{sect:mult-state-models,sect:mult-decr-models,sect:mult-life-models} in
the following three aspects:
\begin{enumerate}[label={(\arabic*)}]
\item \emph{Probabilistic calculations:} We will discuss formulas of various
probabilistic quantities for the new models.

\item \emph{Insurance and annuity EPV calculations:} Armed with the formulas
for computing probabilistic quantities, we will introduce EPV calculation
formulas for different types of \emph{state-contingent} (more general than
``life contingent''!) insurances and annuities. As you will see, the general
EPV calculation formula introduced in STAT3901 is still very useful
\faIcon[regular]{thumbs-up} here!

\item \emph{Premium and policy value calculations:} With the knowledge about
insurance and annuity EPVs, we will explore premium and policy value
calculations under these new models, and also calculations of some
probabilistic quantities about the loss random variables underlying the
premiums and policy values.
\end{enumerate}
\end{enumerate}
\subsection{Probabilistic Calculations}
\begin{enumerate}
\item \textbf{Motivation.} In STAT3901, we have exclusively distinguished lives
according to their survival status: A life is either \emph{alive} or
\emph{dead}. However, real life is not that binary. For instance, a life can be
healthy \faIcon{child}, sick \faIcon{head-side-mask}, disabled
\faIcon{wheelchair}, retired, etc. Intuitively, these kinds of lives can have
rather different ``characteristics'', so it makes sense to model differently
for lives in different \emph{states}.  Mathematically, we are going to use a
\emph{multiple state model} to do this.

\item The simplest multiple state model is the one we have always been studying
in STAT3901, which is called the \defn{alive-dead model}. It can be graphically
represented as follows.
\begin{center}
\begin{tikzpicture}
\node[draw, minimum width=3cm, minimum height=1cm] () at (1.5,0.5) {Alive (0)};
\node[draw, minimum width=3cm, minimum height=1cm] () at (6.5,0.5) {Dead (1)};
\draw[-Latex] (3,0.5) -- (5,0.5)
node[midway, above]{die};
\end{tikzpicture}
\end{center}
The model has two \emph{states}: alive (labelled as ``state 0'') and dead
(labelled as ``state 1''). The rightward arrow suggests the possible direction
of transitions in states: Certainly it is only possible to go/transit from
state 0 to 1, but not the reverse!

\item \textbf{Terminologies.} A multiple state model is:
\begin{itemize}
\item \defn{discrete-time} if transition can only take place at the end of
periods (often years).
\item \defn{continuous-time} if transition can take place at any time.
\end{itemize}

\item \textbf{Notations.} We need some new probability notations for multiple
state model.
\begin{center}
\begin{tabular}{ll}
\toprule
Notation&Probability that \((x)\) ... \\
\midrule
\(\px[\vc{t}]{x}[\blc{i}\mgc{j}]\)&\((x)\) is \underline{currently in state
\blc{\(i\)}} and \underline{will be in state \mgc{\(j\)} at time \vc{\(t\)}}. \\
\midrule
\(\px[\vc{t}]{x}[\overline{\blc{ii}}]\)&
\makecell[l]{
\((x)\) is \underline{currently in state \blc{\(i\)}} and will \underline{stay
in state \blc{\(i\)} for \(\vc{t}\) years}\\ (i.e.,
without leaving state \blc{\(i\)} in between).} \\
\bottomrule
\end{tabular}
\end{center}
\begin{remark}
\item The notations here are not standard actuarial notations.
\item \(\px[t]{x}[ij]\) and \(\px[t]{x}[\overline{ii}]\) are called
\defn{transition probability} and \defn{occupancy probability} respectively.
\item The ``\(t\)''s in the notations can be dropped when \(t=1\).
\end{remark}

Examples: In the alive-dead model,
\begin{itemize}
\item \(\px[10]{30}[01]\) is the probability that a life aged 30 will be dead
at time \(10\). Using the notations in STAT3901, it is ``\(\qx[10]{30}\)''.
\item \(\px[10]{30}[00]=\px[10]{30}[\overline{00}]\) is the probability that a
life aged 30 will be alive at time \(10\) (equivalently, be always alive for 10
years). Using the notations in STAT3901, it is ``\(\px[10]{30}\)''.

\begin{note}
Here the occupancy probability may not appear to be useful, as we are working
in a multiple state model where it is not possible to go back from state \(1\)
to state \(0\), so \(\px[t]{x}[00]=\px[t]{x}[\overline{00}]\) always. But as we
will see, in some more advanced multiple state models, the occupancy
probability \(\px[t]{x}[\overline{ii}]\) can be \emph{different} from
\(\px[t]{x}[ii]\).
\end{note}
\end{itemize}
\item \label{it:gen-prob-fmla} \textbf{General probability calculation
formula.} Like the general EPV calculation formula we learn in STAT3901, there
is a general \emph{probability} calculation formula available, which is helpful
for dealing with probability calculations for multiple state model in general.
\[
\boxed{\text{Desired probability}=\underset{\text{all desired transition paths}}{\sum\text{ or }\int}
\text{probability of going through the path}}.
\footnote{\emph{(If you are interested)} The justification of this formula is
related to the mathematical formulation of multiple state model as a
\emph{Markov chain} (STAT3903). For more details about the mathematical
formulation, see \textcite{dickson2019actuarial}.}
\]

\subsubsection*{Discrete-Time Multiple State Models}
\item \label{it:chapman-kolmogorov} To illustrate the general probability calculation formula, we consider a
special case that is helpful for probability computations in discrete-time
multiple state models. It is known as the \defn{Chapman-Kolmogorov equation}:
For any states \blc{\(i\)} and \mgc{\(j\)},
\[
\px[t+s]{x}[\blc{i}\mgc{j}]=\sum_{\text{all states \(k\)}}^{}\px[t]{x}[\blc{i}k]\times \px[s]{x+t}[k\mgc{j}],
\]
where \(t\) and \(s\) are any nonnegative integers.
\begin{center}
\begin{tikzpicture}
\node[] () at (0,5) {Time \(0\)};
\node[] () at (3,5) {Time \(t\)};
\node[] () at (6,5) {Time \(t+s\)};
\node[draw] (start) at (0,2) {State \blc{\(2\)}};
\node[draw] (end) at (6,3) {State \mgc{\(3\)}};
\foreach \x in {0,...,4} {
\node[draw] () at (3,\x) {State \(\x\)};
\draw[-Latex] (start) -- node[midway]{\(\px[t]{x}[\blc{2}\x]\)} (2.3,\x);
\draw[-Latex] (3.7,\x) -- node[midway]{\(\px[s]{x+t}[\x\mgc{3}]\)} (end);
}
\end{tikzpicture}
\end{center}
From this picture, we can see that summing all the states \(k\) would involve
all the paths desired. Each term \(\px[t]{x}[\blc{i}k]\times
\px[s]{x+t}[k\mgc{j}]\) in the sum corresponds to the probability of going
through a path.

\item If you still recall the linear algebra knowledge from MATH1821/MATH2822
(perhaps not), you may recognize that the sum is somewhat similar to a matrix
multiplication. In fact, we can indeed express the Chapman-Kolmogorov equation
in a matrix form.


First we define the \defn{\(k\)-step transition matrix} (\(k\) is a positive
integer) for a multiple state model with \(n+1\) states \(0,1,\dotsc,n\) as the
following \((n+1)\times (n+1)\) matrix:
\[
P^{(k)}=\mqty[
\px[k]{x}[00]&\px[k]{x}[01]&\cdots&\px[k]{x}[0n] \\
\px[k]{x}[10]&\px[k]{x}[11]&\cdots&\px[k]{x}[1n] \\
\vdots&\vdots&\ddots&\vdots \\
\px[k]{x}[n0]&\px[k]{x}[n1]&\cdots&\px[k]{x}[nn]
]
\]
\begin{note}
Usually we just call a \(1\)-step transition matrix simply as \defn{transition
matrix} and denote it by \(P\).
\end{note}
Then, we can express the Chapman-Kolmogorov equation in the following matrix
form:
\[
P^{(t+s)}=P^{(t)}P^{(s)}.
\]
where \(P^{(t)}P^{(s)}\) denotes the matrix product of \(P^{(t)}\) and
\(P^{(s)}\). Particularly, it implies that the \(k\)-step transition matrix is
indeed the \(k\)th power of the (\(1\)-step) transition matrix:
\(P^{(k)}=P^k\).

\item \label{it:first-step-analysis} \textbf{First step analysis.} Apart from
the Chapman-Kolmogorov equation, another useful tool for calculating
probabilistic quantities in discrete-time multiple state models is the so-called
\emph{first-step analysis}. Essentially it is an application of the conditional
variant of law of total probability/expectation:
\begin{itemize}
\item \emph{(probability)}
\[
\boxed{p_{ij}:=\prob{\text{ever visit state \(j\)}|\text{start in state \(i\)}}
=\sum_{\text{all states \(k\)}}^{}\prob{\text{\gc{first step} \(i\to k\)}}\times p_{kj}}.
\]
\item \emph{(expectation)}
\begin{align*}
\boxed{e_{ij}}&:=\expv{\text{time spent in state \(j\)}|\text{start in state \(i\)}} \\
&=\boxed{\sum_{\text{all states \(k\)}}^{}\prob{\text{\gc{first step} \(i\to k\)}}
\qty(e_{kj}+\delta_{ij})}
\end{align*}
where \(\delta_{ij}\) equals \(1\) if \(i=j\), and \(0\) otherwise.
\begin{note}
\(\delta_{ij}\) is included to count also the first period spent in state
\(i\), in case when \(i=j\). Note that \(e_{kj}\) in the expression is
measuring the mean time spent in state \(j\) \emph{after} the first period,
when the life is in state \(k\) at the start of the second period (time \(1\)).
\end{note}
\end{itemize}
Here \(\prob{\text{\gc{first step} \(i\to k\)}}\) refers to the transition
probability \(\px[]{x}[ik]\).

The probability formula is actually an instance of the general
probability calculation formula in \labelcref{it:gen-prob-fmla} again.
\begin{center}
\begin{tikzpicture}
\node[] () at (0,5) {Time \(0\)};
\node[] () at (3,5) {Time \(1\)};
\node[draw] (start) at (0,2) {State \blc{\(2\)}};
\node[draw, circle] (end) at (11,3) {Ever visit state \mgc{\(3\)}};
\foreach \x in {0,...,4} {
\node[draw] () at (3,\x) {State \(\x\)};
\draw[-Latex] (start) -- node[midway]{\(\px{x}[\blc{2}\x]\)} (2.3,\x);
\draw[-Latex] (3.7,\x) -- node[midway]{\(p_{\x\mgc{3}}\)} (end);
}
\end{tikzpicture}
\end{center}
For the expectation formula, it can be understood via a similar intuition:
\begin{center}
\begin{tikzpicture}
\node[] () at (0,5) {Time \(0\)};
\node[] () at (3,5) {Time \(1\)};
\node[draw] (start) at (0,2) {State \blc{\(2\)}};
\foreach \x in {0,...,4} {
\node[draw] () at (3,\x) {State \(\x\)};
\draw[-Latex] (start) -- node[midway]{\(\px{x}[\blc{2}\x]\)} (2.3,\x);
\draw[-Latex, magenta] (3.7,\x) -- node[midway, above, black]{\(e_{\x\mgc{2}}\)} (11,\x);
}
\draw[-Latex, magenta] (start) --node[midway, below, text=black, text opacity=1, fill=yellow, fill opacity=0.2]{\(\delta_{\blc{2}\mgc{2}}=1\)} (2.3,2);
\end{tikzpicture}
\end{center}
\subsubsection*{Continuous-Time Multiple State Models}
\item \textbf{Force of transition.} The concept of \emph{force of transition}
is exclusive to \emph{continuous-time} multiple state models. As one may
expect, it has a certain similarity to the \emph{force of mortality} learnt in
STAT3901.

Recall from STAT3901 that the force of mortality \(\mu_{x}\) is given by
\[
\mu_{x}=\frac{\displaystyle -\dv{}{x}\px[x]{0}}{\px[x]{0}}
=\lim_{h\to 0^{+}}\frac{1}{\px[x]{0}}\frac{\px[x]{0}-\px[x+h]{0}}{h}
=\lim_{h\to 0^{+}}\frac{1-\px[h]{x}}{h}
=\lim_{h\to 0^{+}}\frac{\qx[h]{x}}{h}
\overset{\text{(alive-dead model)}}{=}
\lim_{h\to 0^{+}}\frac{\px[h]{x}[01]}{h}.
\]

This leads us to define the \defn{force transition} from state \(i\) to state
\(j\ne i\), for a life aged \(x\), by:
\[
\mu_{x}^{ij}=\lim_{h\to 0^{+}}\frac{\px[h]{x}[ij]}{h}.
\]
Here, like the force of mortality, we can interpret \(\mu_{x}^{ij}\times \Delta
x\) as the approximated probability for \((x)\) to transit from state \(i\) to
state \(j\ne i\) in \(\Delta x\) years, when \(\Delta x\) is small.
Symbolically, we can write \(\px[\Delta x]{x}[ij]\approx \mu_{x}^{ij}\times
\Delta x\) when \(\Delta x\) is small.

\item \label{it:occup-fmla} \textbf{Occupancy probability formula.} In
discrete-time multiple state model, we only have ``nice'' formulas for
transition probability, but not occupancy probability. It turns out that in
\emph{continuous-time} multiple state model, there is a ``nice'' formula
available for occupancy probability, in terms of forces of transition:
\[
\boxed{\px[t]{x}[\overline{ii}]=\exp\qty(-\int_{0}^{t}\sum_{j\rc{\ne i}}^{}\mu_{x+s}^{ij}\dd{s})}.
\]
\begin{remark}
\item This formula is not obtained based on the general probability formula in
\labelcref{it:gen-prob-fmla}, so it deserves some more attention.
\item Intuitively, \(\sum_{j\rc{\ne i}}^{}\mu_{x+s}^{ij}\) sums up all the
forces ``driving'' the life to leave state \(i\). The larger the sum, the
smaller the occupancy probability \(\px[t]{x}[\overline{ii}]\).
\item \emph{(If you are interested)} The derivation of the occupancy
probability formula involves some regularity assumptions on continuous-time
multiple state model, which are omitted (and implicitly imposed) here. For the
purpose of STAT3909, it is more important to know when and how to apply the
formulas, and their proofs are not required. Nonetheless, in case you are still
interested in the omitted assumptions and the derivations, see
\textcite{dickson2019actuarial}.
\end{remark}

The occupancy probability formula can be seen as a generalization to the
following \(\mu_{x+t}\to \px[t]{x}\) formula from STAT3901:
\[
\px[t]{x}=\exp\qty(-\int_{0}^{t}\mu_{x+s}\dd{s}).
\]
Actually, this is a special case of the occupancy probability formula here
since we can write
\[
\px[t]{x}[\overline{00}]=\exp\qty(-\int_{0}^{t}\mu_{x+s}^{01}\dd{s})
\]
using the alive-dead model notations.
\item \label{it:kolmogorov-fwd-eq}
\textbf{Kolmogorov\footnote{\href{https://en.wikipedia.org/wiki/Andrey_Kolmogorov}{The
same Kolmogorov} as the one in Chapman-Kolmogorov equation!} forward equation.}
Like \emph{Thiele's differential equation} in
\Cref{subsect:policy-val-cts-recurs}, \emph{Kolmogorov forward equation} is
also a differential equation, but it is about \emph{transition probabilities}.
Nonetheless, it admits a similar form of intuitive interpretation as Thiele's
differential equation.

First we state the \defn{Kolmogorov forward equation}:
\[
\boxed{\dv{}{t}\px[t]{x}[ij]=\sum_{\text{state }k\ne j}^{}\qty(\px[t]{x}[ik]\mu_{x+t}^{kj}-\px[t]{x}[ij]\mu_{x+t}^{jk})}
\]
for any states \(i\) and \(j\).\footnote{\emph{(If you are interested)} To
learn more about the derivation of Kolmogorov forward equation, see
\textcite{dickson2019actuarial}.}

Again, to understand it more intuitively, we integrate both sides from \(0\) to
a positive value \(s\):
\[
\px[s]{x}[ij]-\px[0]{x}[ij]=
\int_{0}^{s}\sum_{\text{state }k\ne j}^{}\qty(\px[t]{x}[ik]\mu_{x+t}^{kj}-\px[t]{x}[ij]\mu_{x+t}^{jk})\dd{t}.
\]
After some rearrangements and slight abuse of notations, we get
\[
\px[s]{x}[ij]=\underbrace{\px[0]{x}[ij]}_{\delta_{ij}}+
\int_{0}^{s}\gc{\sum_{k\ne j}^{}\px[t]{x}[ik]\mu_{x+t}^{kj}\dd{t}}
-\rc{\sum_{k\ne j}^{}\px[t]{x}[ij]\mu_{x+t}^{jk}\dd{t}}.
\]
From here we can see that there are two \emph{opposing} ``forces''
\faIcon{compress-alt} active between time \(0\) and \(s\) that influence the
chance for the life to be in state \(j\) after \(s\) years. More specifically,
we can interpret it as follows:
\begin{enumerate}[label={(\arabic*)}]
\item ``Starting'' probability: \(\px[0]{x}[ij]\), which serves as a baseline
on which changes are continuously made to finally result in the \(s\)-year
transition probability \(\px[s]{x}[ij]\).

The probability is \(1\) if \(i=j\) and \(0\) otherwise. Intuitively, this
means that if the life \emph{starts} at state \(j\), then there is a somewhat
large ``bonus'' on the chance for the life to be in state \(j\) after \(s\)
years (the ``starting line'' is at \(1\)).

\item In every ``infinitesimal'' time interval \([t,t+\dd{t}]\) between time
\(0\) and time \(s\):
\begin{itemize}
\item \faIcon{plus-circle} \gc{\emph{(incoming probability)}} \(\sum_{k\ne j}^{}
\underbrace{\px[t]{x}[ik]}_{\text{\(i\to k\) after \(t\) years}}\times 
\underbrace{\mu_{x+t}^{kj}\dd{t}}_{\text{\gc{enter} state \(j\) in \([t,t+\dd{t}]\)}}\)
\item \faIcon{minus-circle} \rc{\emph{(outgoing probability)}}
\(\sum_{k\ne j}^{}
\underbrace{\px[t]{x}[ij]}_{\text{\(i\to j\) after \(t\) years}}\times 
\underbrace{\mu_{x+t}^{jk}\dd{t}}_{\text{\rc{leave} state \(j\) \([t,t+\dd{t}]\)}}\)
\end{itemize}
In short, the \gc{incoming probability} (\rc{outgoing probability}) sums up the
probabilities corresponding to all possible paths for \gc{entering}
(\rc{leaving}) state \(j\) in the time interval \([t,t+\dd{t}]\), and we are
\underline{adding} \gc{incoming probability} and \underline{subtracting}
\rc{outgoing probability}. Note that \defn{entering state} \(j\) means
transiting to state \(j\) from a \emph{different} state.
\item ``Summing'' up all these ``infinitesimal'' contributions to the value
\(\px[0]{x}[ij]\), we get the ``ending'' (final) probability:
\(\px[s]{x}[ij]\).
\end{enumerate}
\item \label{it:gen-tran-prob-fmla} \textbf{General transition probability
formula.} Lastly, let us consider a general transition probability formula in
continuous-time multiple state model, which is based on the general probability
calculation formula in \labelcref{it:gen-prob-fmla}: For any states \(i\) and
\(j\),
\[
\px[t]{x}[ij]=\int_{0}^{t}\qty(\sum_{k\ne j}^{}\px[s]{x}[ik]\mu_{x+s}^{kj})
\underbrace{\px[t-s]{x+s}[\overline{jj}]}_{\mathclap{\text{don't miss \warn{}}}}\dd{s}.
\footnote{
\emph{(If you are interested)} This formula is indeed obtained by solving the
Kolmogorov forward equation and applying the occupancy probability formula.
Solving the Kolmogorov forward equation is manageable (but not too simple
still) since it can be rewritten in the form
\[
\dv{}{t}\vc{\px[t]{x}[ij]}+\qty(\sum_{k\ne j}^{}\mu_{x+t}^{jk})\vc{\px[t]{x}[ij]}
=\sum_{k\ne j}^{}\px[t]{x}[ik]\mu_{x+t}^{kj},\]
which is a first order linear ordinary differential equation (ODE) (learnt in
MATH1821). Probably you already forget what this means and how to solve this
kind of ODE, but it turns out that this form of ODE can be solved using the
method of ``integration factor''. (Frankly speaking, I have to look this up
also... \faIcon[regular]{grin-beam-sweat})
}
\]
This seemingly complicated formula is indeed an instance of the general
probability calculation formula.

To see this, we first abuse notations a bit and write
\[
\px[t]{x}[ij]=\int_{0}^{t}\qty(\sum_{k\ne j}^{}\px[s]{x}[ik]\mu_{x+s}^{kj}\dd{s})\px[t-s]{x+s}[\overline{jj}].
\]
Then, for every time point \(s\) between time \(0\) and time \(t\),
\begin{itemize}
\item  \(\sum_{k\ne j}^{} \underbrace{\px[s]{x}[ik]}_{\text{\(i\to k\) after
\(s\) years}}\times \underbrace{\mu_{x+s}^{kj}\dd{s}}_{\text{\gc{enter} state
\(j\) in \([s,t+\dd{s}]\)}}\)
is the probability
for \gc{entering} state \(j\) in the time interval \([s,s+\dd{s}]\) (the
``incoming probability'' in the intuitive interpretation of Kolmogorov forward
equation).
\item \(\px[t-s]{x+s}[\overline{jj}]\) is the probability for the life to
\emph{stay} in the state \(j\) for the remaining \(t-s\) years (i.e., until
time \(t\)).
\end{itemize}
Hence, \(\qty(\sum_{k\ne
j}^{}\px[s]{x}[ik]\mu_{x+s}^{kj}\dd{s})\px[t-s]{x+s}[\overline{jj}]\) is the
probability of being in the state \(j\) \faIcon{at} time \(t\), by entering
state \(j\) in the time interval \([s,s+\dd{s}]\) and then staying in the state
\(j\) until time \(t\).  ``Summing'' over every time point \(s\) between time
\(0\) and time \(t\) would then involve all the paths for being in the state
\(j\) \faIcon{at} time \(t\).

\begin{note}
For paths which enter the state \(j\) multiple times between time \(0\) and
time \(t\), they would be covered by the probability expression for time point
\(s\) which equals the \emph{last} time of entry to the state \(j\).
\end{note}
\begin{center}
\begin{tikzpicture}
\node[] () at (0,5) {Time \(0\)};
\node[] () at (2,5) {Time \(t_1\)};
\node[] () at (6,5) {Time \(t_2\)};
\node[violet, draw] () at (9,5) {Time \(s\)};
\node[] () at (12,5) {Time \(t\)};
\node[draw] () at (0,2) {State \blc{\(2\)}};
\node[draw] () at (2,3) {State \(3\)};
\node[draw] () at (6,1) {State \(1\)};
\node[draw] () at (9,3) {State \(3\)};
\node[draw] () at (12,3) {State \mgc{\(3\)}};
\draw[-Latex, orange] (0.8,2) -- (2,2) -- (2,2.7);
\draw[-Latex, orange] (2.8,3) -- (6,3) -- (6,1.3);
\draw[-Latex, orange] (6.8,1) -- (9,1);
\draw[-Latex, violet] (9,1) --node[midway, right]{\(\mu_{x+s}^{13}\dd{s}\)} (9,2.7);
\draw[-Latex, brown!50!black] (9.8,3) --node[midway, above]{\(\px[t-s]{x+s}[\overline{33}]\)} node[midway, below]{stay} (11.2,3);
\draw[very thick, decorate,decoration={mirror, calligraphic brace, amplitude=5pt, raise=5pt}] (0,0.5) -- (9,0.5)
node[midway, below=0.3cm, orange]{\(\px[s]{x}[21]\)};
\end{tikzpicture}
\end{center}
Several specialized and simplified versions of this general transition
probability formula will appear in
\Cref{sect:mult-decr-models,sect:mult-life-models}.
\end{enumerate}
\subsection{Insurance and Annuity EPV Calculations}
\label{subsect:mult-state-insur-ann-epv}
\begin{enumerate}
\item After having the (somewhat lengthy) discussion about probabilistic
calculations in multiple state models, we are now prepared for calculating
insurance and annuity EPVs. Here we avoid using the terms ``\underline{life}
insurance'' and ``\underline{life} annuity'' because the insurance and annuity
in multiple state model are \emph{not necessarily} \underline{life} contingent
anymore. We can only say that they are \underline{state} contingent in general.

\item \textbf{Applications of multiple state model.} An
\emph{insurance} here may provide a benefit\footnote{Note that we do not use
the term ``death benefit'' or ``survival/endowment benefit'' here!} at the end
of year of each entry to a state \(k\). Depending on what state \(k\)
represents, the insurance can be interpreted differently, e.g.:
\begin{itemize}
\item \emph{\(\text{State \(k\)}=\text{Dead}\)}: Life insurance
\item \emph{\(\text{State \(k\)}=\text{Permanently Disabled}\):} Disability insurance
\item \emph{\(\text{State \(k\)}=\text{Sick}\):} Health insurance
\item \(\vdots\)
\end{itemize}
\begin{center}
\begin{tikzpicture}
\node[draw, minimum width=3cm, minimum height=1cm] () at (0,0) {Healthy (0)};
\node[draw, minimum width=3cm, minimum height=1cm] () at (6,0) {Sick (1)};
\node[draw, minimum width=3cm, minimum height=1cm] () at (3,-2) {Dead (2)};
\draw[-Latex] (1.5,0.3) -- (4.5,0.3);
\draw[-Latex] (4.5,-0.3) -- (1.5,-0.3);
\draw[-Latex] (0,-0.5) -- (2.5,-1.5);
\draw[-Latex] (6,-0.5) -- (3.5,-1.5);
\end{tikzpicture}
\end{center}

Likewise, an \emph{annuity} here may provide level benefit payments at the
start of each year whenever the annuitant is in a state \(j\), and there is a
high flexibility in the nature of annuity, based on the setting of state \(j\).
For example:
\begin{itemize}
\item \emph{\(\text{State \(j\)}=\text{Alive}\):} Life annuity
\item \emph{\(\text{State \(j\)}=\text{Temporarily Disabled}\):} Disability income benefits
\item \emph{\(\text{State \(j\)}=\text{Retired}\):} Pension (more in STAT3956)
\item \(\vdots\)
\end{itemize}
From this, we can see that multiple state model provides a general framework for
working with many types of insurance and annuity products, through ``life
contingencies'' concepts and methods.

\item \textbf{Terminologies.} Like STAT3901, the two main types of
state-contingent insurances and annuities are (i) \emph{discrete} and (ii)
\emph{continuous}, which are defined by:
\begin{itemize}
\item \defn{Discrete insurance}: Benefit is paid at the end of year of
\textbf{each} entry to a state, within policy term.
\item \defn{Continuous insurance}: Benefit is paid at \textbf{each} entry to a state, within policy term.
\item \defn{Discrete annuity-due}: Benefit is paid at the start of each year
whenever the annuitant is in a state, within policy term.
\item \defn{Continuous annuity}: Benefit is paid continuously while the
annuitant is in a state, within policy term.
\end{itemize}
\begin{warning}
There can be \emph{multiple} insurance benefit payments in multiple state
model, unlike the case in STAT3901!
\end{warning}

For simplicity, here we will not deal with the \(1/m\)thly case, although
insurance and annuity EPVs in such case can be developed in a similar fashion.

\item \textbf{EPV notations.} Like STAT3901, we have some notations for EPV of
various insurances and annuities. Again, we shall assume that the amount of
each benefit payment (for discrete case) or the rate of benefit payment (for
continuous case) is \(1\).

Comparing with the EPV notations in STAT3901, the EPV notations here have extra
\emph{superscripts} which carry information about the \emph{states} involved.
For example, the notation \(\Ax{x}[\blc{i}\mgc{k}]\) denotes the EPV of
discrete whole life insurance for a life aged \(x\) \underline{currently in
state \blc{\(i\)}}, which pays a benefit of \(1\) at the end of year of
\textbf{each} \underline{entry to state \mgc{\(k\)}}.

\item \label{it:mult-state-epv-fmlas} \textbf{EPV formulas.} The EPV formulas can all be developed using the
general EPV calculation formula we learn in STAT3901, demonstrating its high
utility \faIcon[regular]{thumbs-up}:
\[
\text{EPV}=\underset{\text{all possible payment times}}{\sum\text{ or }\int}\brc{\text{benefit amount}}\times\vc{\text{discount factor}}\times\mgc{\text{prob.\ of triggering payment}}.
\]

Based on this general formula, we can obtain the following more frequently used
EPV formulas:
\begin{center}
\begin{tabular}{lll}
\toprule
Type&Discrete&Continuous \\
\midrule
Permanent insurance&
\(\displaystyle \Ax{x}[ik]=\sum_{t=0}^{\infty}\vc{v^{t+1}}\mgc{\qty(\sum_{j\ne k}^{}\px[t]{x}[ij]\px{x+t}[jk])}\)
&\(\displaystyle \Ax*{x}[ik]=\int_{0}^{\infty}\vc{e^{-\delta t}}\mgc{\sum_{j\ne k}^{}\px[t]{x}[ij]\mu_{x+t}^{jk}\dd{t}}\) \\
Permanent annuity&
(due) \(\displaystyle \ax**{x}[ij]=\sum_{t=0}^{\infty}\vc{v^{t}}\mgc{\px[t]{x}[ij]} \)
& \(\displaystyle \ax*{x}[ij]=\int_{0}^{\infty}\vc{e^{-\delta t}}\mgc{\px[t]{x}[ij]}\brc{\dd{t}} \)
\\
\(n\)-year term insurance&
\(\displaystyle \Ax{\endowxn}[ik]
=\sum_{t=0}^{\blc{n-1}}\vc{v^{t+1}}\mgc{\qty(\sum_{j\ne k}^{}\px[t]{x}[ij]\px{x+t}[jk])}\)
&\(\displaystyle \Ax*{\endowxn}[ik]
=\int_{0}^{\blc{n}}\vc{e^{-\delta t}}\mgc{\sum_{j\ne k}^{}\px[t]{x}[ij]\mu_{x+t}^{jk}\dd{t}}\) \\
\(n\)-year temporary annuity&
(due) \(\displaystyle \ax**{\endowxn}[ij]=\sum_{t=0}^{\blc{n-1}}\vc{v^{t}}\mgc{\px[t]{x}[ij]}\)
& \(\displaystyle \ax*{\endowxn}[ij]=\int_{0}^{\blc{n}}\vc{e^{-\delta t}}\mgc{\px[t]{x}[ij]}\brc{\dd{t}} \)
\\
\bottomrule
\end{tabular}
\end{center}
\begin{remark}
\item Recall that \emph{entering} a state \(k\) means transiting to state \(k\) from
a \emph{different} state!
\item There is not a ``\(1\)'' on top of \(x\) in the  notations for \(n\)-year term
insurance, to avoid making the notations clumsy. Note that in a general
multiple state model (not necessarily alive-dead model), we do not have a
notation for denoting the EPV of an ``\(n\)-year endowment insurance''. We do
have ``annuity-insurance formula'' either.

In general, for calculating EPV of an ``unfamiliar'' policy (not in the list
above), we need to start with the general EPV calculation formula above.

\item Since multiple insurance benefit payments are possible, it is possible
for the insurance EPV to \emph{exceed \(1\)}, unlike the case in STAT3901.
\end{remark}


\item \label{it:mult-state-epv-recursive-fmlas} \textbf{Recursive formulas for EPVs.} Apart from the EPV calculation
formulas above, another type of formula that is of interest is the
\emph{recursive formulas} for EPVs. The good \faIcon[regular]{thumbs-up} thing
is that the recursive formulas here do have similar form as the ones we learn
in STAT3901. But the ``bad'' \faIcon[regular]{thumbs-down} thing is that care
must be taken when writing down the recursive formulas, to ensure that the
transitions between states are correctly handled. We can't simply write down
``\(\Ex[n]{x}\)'' without thought anymore! Nonetheless, this should not be an issue if
you can gain an \emph{intuitive} understanding of the recursive formulas below.
\begin{note}
In the following, \(\sum_{j}^{}\) denotes the sum over every state \(j\).
\end{note}
\begin{itemize}
\item \emph{(more commonly used) Permanent insurance:}
\begin{itemize}
\item \emph{(discrete)} \(\Ax{x}[\blc{i}\mgc{k}]
=\Ax{\endowxn}[\blc{i}\mgc{k}]+\sum_{j}^{}v^{n}\px[n]{x}[\blc{i}j]\Ax{x+n}[j\mgc{k}]
\overset{(n=1)}{=}v\px[]{x}[\blc{i}\mgc{k}]+\sum_{j}^{}v\px[]{x}[\blc{i}j]\Ax{x+1}[j\mgc{k}]\).
\item \emph{(continuous)} \(\Ax*{x}[\blc{i}\mgc{k}]=\Ax*{\endowxn}[\blc{i}\mgc{k}]
+\sum_{j}^{}v^{n}\px[n]{x}[\blc{i}j]\Ax*{x+n}[j\mgc{k}]\).
\end{itemize}
\begin{center}
\begin{tikzpicture}
\draw[-Latex] (0,-1) -- (10,-1) node[right]{Time};
\fill[] (0,-1) circle [radius=0.05]
node[below] {0}
node[above=0.1cm] {\((x)\)};
\node[draw] (start) at (0,0.5) {State \blc{\(i\)}};
\fill[] (3,-1) circle [radius=0.05]
node[below] {\(n\)};
\draw[->] (0,-1.5) --node[midway, below]{\(\Ax*{\endowxn}[\blc{i}\mgc{k}]\)} (3,-1.5);
\foreach \x in {0,...,2} {
\node[draw] () at (3,\x) {State \x};
\draw[->] (3,\x) --node[midway, above]{\(\Ax*{x+n}[\x\mgc{k}]\)} (9.5,\x);
\draw[-Latex, color=ForestGreen] (3,\x) to[bend right] node[midway, black] {\(\times v^{n}\px[n]{x}[\blc{i}\x]\)} (0,0.6);
}
\end{tikzpicture}
\end{center}
\item \emph{(more commonly used) Permanent annuity:}
\begin{itemize}
\item \emph{(discrete, due)} \(\ax**{x}[\blc{i}\mgc{k}]
=\ax**{\endowxn}[\blc{i}\mgc{k}]+\sum_{j}^{}v^{n}\px[n]{x}[\blc{i}j]\ax**{x+n}[j\mgc{k}]
\overset{(n=1)}{=}\delta_{\blc{i}\mgc{k}}+\sum_{j}^{}v\px[]{x}[\blc{i}j]\ax**{x+1}[j\mgc{k}]\).
\item \emph{(continuous)} \(
\ax*{x}[\blc{i}\mgc{k}]
=\ax*{\endowxn}[\blc{i}\mgc{k}]+\sum_{j}^{}v^{n}\px[n]{x}[\blc{i}j]\ax*{x+n}[j\mgc{k}]
\).
\end{itemize}
\begin{center}
\begin{tikzpicture}
\draw[-Latex] (0,-1) -- (10,-1) node[right]{Time};
\fill[] (0,-1) circle [radius=0.05]
node[below] {0}
node[above=0.1cm] {\((x)\)};
\node[draw] (start) at (0,0.5) {State \blc{\(i\)}};
\fill[] (3,-1) circle [radius=0.05]
node[below] {\(n\)};
\draw[->] (0,-1.5) --node[midway, below]{\(\ax*{\endowxn}[\blc{i}\mgc{k}]\)} (3,-1.5);
\foreach \x in {0,...,2} {
\node[draw] () at (3,\x) {State \x};
\draw[->] (3,\x) --node[midway, above]{\(\ax*{x+n}[\x\mgc{k}]\)} (9.5,\x);
\draw[-Latex, color=ForestGreen] (3,\x) to[bend right] node[midway, black] {\(\times v^{n}\px[n]{x}[\blc{i}\x]\)} (0,0.6);
}
\end{tikzpicture}
\end{center}
\item \emph{\(n\)-year term insurance:}
\begin{itemize}
\item \emph{(discrete)} \(\Ax{\endowxn}[\blc{i}\mgc{k}]
\overset{(m\le n)}{=}\Ax{x:\angl{m}}[\blc{i}\mgc{k}]
+\sum_{j}^{}v^{m}\px[m]{x}[\blc{i}j]\Ax{x+m:\angl{n-m}}[j\mgc{k}]
\overset{(m=1)}{=}v\px[]{x}[\blc{i}\mgc{k}]+\sum_{j}^{}v\px[]{x}[\blc{i}j]\Ax{x+1:\angl{n-1}}[j\mgc{k}]\).
\item \emph{(continuous)} \(\Ax*{\endowxn}[\blc{i}\mgc{k}]
\overset{(m\le n)}{=}\Ax*{x:\angl{m}}[\blc{i}\mgc{k}]
+\sum_{j}^{}v^{m}\px[m]{x}[\blc{i}j]\Ax*{x+m:\angl{n-m}}[j\mgc{k}]\).
\end{itemize}
\item \emph{\(n\)-year temporary annuity:}
\begin{itemize}
\item \emph{(discrete, due)} \(\ax**{\endowxn}[\blc{i}\mgc{k}]
\overset{(m\le n)}{=}\ax**{x:\angl{m}}[\blc{i}\mgc{k}]+\sum_{j}^{}v^{m}\px[m]{x}[\blc{i}j]\ax**{x+m}[j\mgc{k}]
\overset{(m=1)}{=}\delta_{\blc{i}\mgc{k}}+\sum_{j}^{}v\px[]{x}[\blc{i}j]\ax**{x+1:\angl{n-1}}[j\mgc{k}]\).
\item \emph{(continuous)} \(
\ax*{\endowxn}[\blc{i}\mgc{k}]
\overset{(m\le n)}{=}\ax*{x:\angl{m}}[\blc{i}\mgc{k}]
+\sum_{j}^{}v^{m}\px[m]{x}[\blc{i}j]\ax*{x+m:\angl{n-m}}[j\mgc{k}]
\).
\end{itemize}
\end{itemize}
\item \textbf{Waiting period.} For annuity benefits like disability income
benefits, the benefits are often not paid \emph{immediately} at the onset of
the disability (the event covered). The time between the event onset and the
start of benefit payments is called the \defn{waiting period}. One reason for
having waiting period is to give the insurer some time to verify the
policyholder's eligibility for the benefit payments (e.g., check
\faIcon{search} the documents to see whether he is really disabled).

With the presence of waiting period, there would be a time delay between the
time of entry to the target state (e.g., temporarily disabled) and the start of
the benefit payments. So, we cannot use the ordinary annuities and formulas
discussed previously to perform calculations about the benefit payments of this
kind of products.

\item \textbf{Continuous sojourn annuity.} To deal with waiting periods for
continuous annuities in continuous-time multiple state model, we need to
introduce a new concept called \emph{continuous sojourn annuity}. It is a
special annuity designed mainly for theoretical purpose (such kind of annuity
can hardly be found in practice). The idea is that it serves as a
``one-time-only'' \underline{continuous} annuity, lasting only for a
\underline{sojourn} (``short staying period'').

A \defn{continuous sojourn annuity} issued to \((x)\) currently in state \(j\)
pays benefits (e.g., at rate \(1\)) continuously while \((x)\) stays in state
\(j\), within policy term; Once \((x)\) leaves state \(j\), the benefits cease
\textbf{forever}, i.e., there would not be any benefits anymore, even if
\((x)\) later goes back to state \(j\).

For such continuous sojourn annuity with benefit rate \(1\), the notations are:
\begin{itemize}
\item \emph{permanent version:} \(\ax*{x}[\overline{jj}]\).
\item \emph{\(n\)-year temporary version:} \(\ax*{\endowxn}[\overline{jj}]\).
\end{itemize}
Applying the general EPV calculation formula, we get
\begin{itemize}
\item \(\displaystyle \ax*{x}[\overline{jj}]=\int_{0}^{\infty}\vc{e^{-\delta t}}
\mgc{\px[t]{x}[\overline{jj}]}\brc{\dd{t}}\).
\item \(\displaystyle \ax*{\endowxn}[\overline{jj}]=\int_{0}^{n}\vc{e^{-\delta t}}
\mgc{\px[t]{x}[\overline{jj}]}\brc{\dd{t}}\).
\end{itemize}

\item \label{it:wait-cts-ann-inej} \textbf{Incorporating waiting periods.}
Using the concept of \emph{continuous sojourn annuity}, we can apply the
general EPV calculation formula (so useful!) to incorporate waiting periods.
First we consider a continuous permanent annuity issued to \((x)\) currently in
state \(i\), with benefit rate \(1\) payable continuously while \((x)\) is in
state \(j\), and waiting period \(w\) years (applicable for \textbf{each} entry
to state \(j\)).

When \rc{\(i\ne j\)}, we have
\[
\boxed{\text{EPV}=\int_{0}^{\infty}\brc{\qty(\ax*{x+t}[\overline{jj}]-\ax*{x+t:\angl{w}}[\overline{jj}])}
\vc{e^{-\delta t}}\mgc{\sum_{k\ne j}^{}\px[t]{x}[ik]\mu_{x+t}^{kj}\dd{t}}}.
\]
To better understand this, consider:
\begin{itemize}
\item ``All possible payment times'' start at time \(0\) with no end, so we have ``\(\int_{0}^{\infty}\)''.
\item \vc{Discount factor} \faIcon{at} time \(t\) is \vc{\(e^{-\delta t}\)}.
\item \mgc{Probability of triggering payment}\footnote{Here the ``payment'' is
indeed possibly zero due to the presence of waiting period.} \faIcon{at} time
\(t\) is loosely \(\mgc{\sum_{k\ne j}^{}\px[t]{x}[ik]\mu_{x+t}^{kj}\dd{t}}\),
which is the probability of \underline{entering} state \(j\) in
\([t,t+\dd{t}]\), which ``generates'' a fresh permanent continuous sojourn
annuity.
\item \brc{Benefit amount} \faIcon{at} time \(t\) is the time-\(t\) EPV of the
``newly generated'' continuous sojourn annuity \faIcon{at} time \(t\), namely
\brc{\(\ax*{x+t}[\overline{jj}]
-\underbrace{\ax*{x+t:\angl{w}}[\overline{jj}]}_{\mathclap{\text{no benefits in the waiting period}}}\)}.

\begin{note}
Since we are computing EPV, it would not make any difference to replace the
continuous stream of payments from the continuous sojourn annuity by a single
payment of amount equal to the EPV of that payment stream \faIcon{at} time \(t\).
This is exactly the replacement we have done here.
\end{note}
\end{itemize}

When we change the policy term to \(n\) years, we can simply change
\(\int_{0}^{\infty}\to\int_{0}^{n}\) in the formula above.
\begin{center}
\begin{tikzpicture}
\node[] () at (0,4.7) {0};
\fill[] (0,5) circle [radius=0.05];
\draw[-Latex] (0,5) -- (13,5) node[right=0.2cm]{Time};
\draw[yellow, opacity=0.5, line width=0.2cm] (3.6,5) -- (6,5);
\draw[yellow, opacity=0.5, line width=0.2cm] (10.6,5) -- (12,5);
\draw[pen colour=blue, very thick, decorate,decoration={calligraphic brace, amplitude=5pt, raise=5pt}] (2,5) -- (3.6,5)
node[midway, above=0.3cm, blue]{waiting period};
\draw[pen colour=blue, very thick, decorate,decoration={calligraphic brace, amplitude=5pt, raise=5pt}] (9,5) -- (10.6,5)
node[midway, above=0.3cm, blue]{waiting period};
\node[draw] () at (0,2) {State \(2\)};
\node[draw] () at (2,3) {State \(3\)};
\node[draw] () at (6,1) {State \(1\)};
\node[draw] () at (9,3) {State \(3\)};
\node[draw] () at (12,0) {State \(0\)};
\draw[-Latex] (0.8,2) -- (2,2) -- (2,2.7);
\draw[-Latex] (2.8,3) -- (6,3) -- (6,1.3);
\draw[-Latex] (6.8,1) -- (9,1) -- (9,2.7);
\draw[-Latex] (9.8,3) -- (12,3) -- (12,0.3);
\end{tikzpicture}
\end{center}


\item \label{it:wait-cts-ann-ieqj} \textbf{Adjustment for \(i=j\) case.}
When \rc{\(i=j\)}, more care is needed \warn{}.  Since the formula above
focuses on continuous sojourn annuities ``newly generated'' at each entry to
state \(j\), it would \rc{miss} the potential payments made immediately after
the first waiting period starting now, which do not require any \emph{entry} to
state \(j\) (as the life is already in state \(j\) at the start). This issue is
illustrated below.
\begin{center}
\begin{tikzpicture}
\node[] () at (0,4.7) {0};
\fill[] (0,5) circle [radius=0.05];
\draw[-Latex] (0,5) -- (13,5) node[right=0.2cm]{Time};
\draw[yellow, opacity=0.5, line width=0.2cm] (1.6,5) -- (2.5,5);
\draw[yellow, opacity=0.5, line width=0.2cm] (5.1,5) -- (6,5);
\draw[yellow, opacity=0.5, line width=0.2cm] (10.6,5) -- (12,5);
\draw[red, dashed] (2.05,5) ellipse [x radius=0.6cm, y radius=0.4cm];
\node[] () at (2.05,4.3) {missed in the formula \warn{}};
\draw[pen colour=blue, very thick, decorate,decoration={calligraphic brace, amplitude=5pt, raise=5pt}] (0,5) -- (1.6,5)
node[midway, above=0.3cm, blue]{waiting period};
\draw[pen colour=blue, very thick, decorate,decoration={calligraphic brace, amplitude=5pt, raise=5pt}] (3.5,5) -- (5.1,5)
node[midway, above=0.3cm, blue]{waiting period};
\draw[pen colour=blue, very thick, decorate,decoration={calligraphic brace, amplitude=5pt, raise=5pt}] (9,5) -- (10.6,5)
node[midway, above=0.3cm, blue]{waiting period};
\node[draw] () at (0,3) {State \(3\)};
\node[draw] () at (2.5,2) {State \(2\)};
\node[draw] () at (3.5,3) {State \(3\)};
\node[draw] () at (6,1) {State \(1\)};
\node[draw] () at (9,3) {State \(3\)};
\node[draw] () at (12,0) {State \(0\)};
\draw[-Latex] (0.8,3) -- (2.5,3) -- (2.5,2.3);
\draw[-Latex] (3.3,2) -- (3.5,2) -- (3.5,2.7);
\draw[-Latex] (4.3,3) -- (6,3) -- (6,1.3);
\draw[-Latex] (6.8,1) -- (9,1) -- (9,2.7);
\draw[-Latex] (9.8,3) -- (12,3) -- (12,0.3);
\end{tikzpicture}
\end{center}
Consequently, to calculate the correct EPV when \(i=j\), we need to add an
extra term \(\brc{\ax*{x}[\overline{jj}]-\ax*{x:\angl{w}}[\overline{jj}]}\),
which accounts for the EPV of the missed potential payments. That is, the
formula becomes:
\begin{itemize}
\item \emph{(permanent annuity)}
\[
\boxed{\text{EPV}=\brc{\ax*{x}[\overline{jj}]-\ax*{x:\angl{w}}[\overline{jj}]}+
\int_{0}^{\infty}\brc{\qty(\ax*{x+t}[\overline{jj}]-\ax*{x+t:\angl{w}}[\overline{jj}])}
\vc{e^{-\delta t}}\mgc{\sum_{k\ne j}^{}\px[t]{x}[ik]\mu_{x+t}^{kj}\dd{t}}}.
\]
\item \emph{(\(n\)-year annuity)}
\[
\boxed{\text{EPV}=\brc{\ax*{x}[\overline{jj}]-\ax*{x:\angl{w}}[\overline{jj}]}+
\int_{0}^{n}\brc{\qty(\ax*{x+t}[\overline{jj}]-\ax*{x+t:\angl{w}}[\overline{jj}])}
\vc{e^{-\delta t}}\mgc{\sum_{k\ne j}^{}\px[t]{x}[ik]\mu_{x+t}^{kj}\dd{t}}}.
\]
\end{itemize}
The formula can still be explained using the terminologies in the general EPV
calculation formula, since we can treat the added term as representing a single
payment made \faIcon{at} time \(0\) with amount equal to
\(\ax*{x}[\overline{jj}]-\ax*{x:\angl{w}}[\overline{jj}]\) (possibly zero). The
discount factor is \vc{\(1\)} and the probability of having this payment is
\mgc{\(1\)} (as the life is known to be in state \(j\) now).
\end{enumerate}
\subsection{Premium and Policy Value Calculations}
\label{subsect:mult-state-prem-pv}
\begin{enumerate}
\item Using the formulas for various insurance and annuity EPVs in
\Cref{subsect:mult-state-insur-ann-epv}, we can compute net/gross premiums via
equivalence principle, like what we did in STAT3901. On the other hand, for
policy values, although the idea behind is still similar to the STAT3901 case,
adjustments are needed to incorporate the effects on policy values from
different \emph{states}. So we will focus on policy values from now on.

\item In STAT3901, the time-\(t\) policy value is defined by
\begin{itemize}
\item \emph{time-\(t\) net premium policy value:}
\[\Vx[t]{}[n]
=\epvt{t}{\text{future benefits}}-\epvt{t}{\text{future \textbf{net} premiums}}.\]
\item \emph{time-\(t\) gross premium policy value:}
\[\Vx[t]{}[g]
=\epvt{t}{\text{future benefits and expenses}}-\epvt{t}{\text{future \textbf{gross} premiums}}.\]
\end{itemize}
Particularly, it is assumed that the policy is still in force \faIcon{at} time
\(t\), i.e., the underlying life is \underline{alive} \faIcon{at} time \(t\).
Using the alive-dead model terminologies, it means that we assume the life is
\underline{in state \(0\)} \faIcon{at} time \(t\).

\item \textbf{Motivation.} For the policies in STAT3901, being in force
\faIcon{at} time \(t\) is equivalent to being \underline{in state \(0\)}
\faIcon{at} time \(t\). But in a more general multiple state model, this is not
necessarily the case. When we just know that the policy is in force \faIcon{at}
time \(t\), there can still be multiple possible states for the life. This
would pose difficulties and complications in our calculations as the formulas
discussed previously depend on a \underline{single} starting state only.

Therefore, to be more compatible with our previous formulas and discussions,
here we shall also define time-\(t\) policy value to be dependent on the
starting state.

\item \textbf{Definition.} The time-\(t\) policy value for life in
\underline{state \(j\)} \faIcon{at} time \(t\) is defined by:
\begin{itemize}
\item \emph{time-\(t\) \defn{net premium policy value}:}
\[
\epvtj{t}{j}{\text{future benefits}}-\epvtj{t}{j}{\text{future \textbf{net} premiums}}.\]
\item \emph{time-\(t\) \defn{gross premium policy value}:}
\[
\epvtj{t}{j}{\text{future benefits and expenses}}-\epvtj{t}{j}{\text{future \textbf{gross} premiums}}.\]
\end{itemize}
Here \(\epvtj{t}{j}{\cdot}\) denotes the time-\(t\) EPV with the starting
(time-\(t\)) state being \underline{state \(j\)}. To avoid complicating the
notations, we usually just use the notation \(\Vx[t]{}[(j)]\) to denote both the
net premium and gross premium policy values. We should be able to tell which is
being considered from the context (usually we are considering gross premium
policy value).

Intuitively, we would expect that \(\Vx[t]{}[(j)]\) could be quite different
for different states \(j\), even if the time \(t\) remains fixed. For instance,
for a whole life insurance, if we set state \(0\) as ``healthy'' \faIcon{child} and state
\(1\) as ``ill'' \faIcon{head-side-mask}, it is natural to expect that
\(\Vx[t]{}[(1)]\) would be higher than \(\Vx[t]{}[(0)]\), since a sick person
\faIcon{head-side-mask} should be more likely to die \faIcon{skull} early than
a healthy person \faIcon{child}, which raises the reserve for sick individual.

\item \label{it:mult-state-basic-policy-val-recurs} \textbf{Basic policy value
recursion.} Like STAT3901, we are interested in developing some recursive
formulas for policy values. To start with, let us consider the basic policy
value version, which involves only annual time points and is for
\emph{discrete} policies.  The ``slogan'' introduced in the STAT3901 study
notes (``what you have is what you need'', or WYHIWYN) still applies, and leads
to the following recursive formula:
\[
\boxed{\underbrace{\qty(\Vx[t]{}[(j)]+G_t^{(j)}-e_t^{(j)})(1+i)}_{\text{what you have}}
=\underbrace{\sum_{\text{all states \(k\)}}^{}\px{x+t}[jk]
\qty(B_{t+1}^{(k)}+E_{t+1}^{(k)}+\Vx[t+1]{}[(k)])}_{\text{what you need (expected)}}}
\]
(\(t\) is an integer time point). Clumsier notations appear here as the
quantities depend on state:
\begin{itemize}
\item \(G_t^{(j)}\): gross premium \faIcon{at} time \(t\), for life
\underline{in state \(j\)} at that time
\item \(e_t^{(j)}\): sum of initial and renewal expenses \faIcon{at} time
\(t\), for life \underline{in state \(j\)} \faIcon{at} that time
\item \(B_{t+1}^{(k)}\): amount of benefits payable \faIcon{at} time \(t+1\) for
life \underline{in state \(k\)} at that time, including: \warn{}
\begin{itemize}
\item \emph{(insurance type benefit)} benefit paid because the life enters state \(k\) in the year \(t+1\)
\item \emph{(annuity type benefit)} benefit paid because the life is in state
\(k\) \faIcon{at} time \(t+1\) (start of year \(t+2\))
\begin{note}
In case there is expense associated to the \underline{annuity type} benefit
\emph{(seldom the case)}, we can incorporate the expense by adding it on top of
the amount of annuity type benefit.
\end{note}
\end{itemize}
\item \(E_{t+1}^{(k)}\): amount of the settlement expense associated to the
\underline{insurance type} benefit payable \faIcon{at} time \(t+1\) for life
\underline{in state \(k\)} at that time
\end{itemize}

This generalizes the basic policy recursive formula \((\Vx[t]{}+G_t-e_t)(1+i)=
\qx{x+t}(S_{t+1}+E_{t+1})+\px{x+t}(\Vx[t+1]{}[])\) (for life insurance)  in
STAT3901, since by considering the alive-dead model, we can express it as
\[
(\Vx[t]{}[(0)]+G_t^{(0)}-e_t^{(0)})(1+i)=\px{x+t}[01](B_{t+1}^{(1)}+E_{t+1}^{(1)}+\underbrace{\Vx[t+1]{}[(1)]}_{0})
+\px{x+t}[00](\underbrace{B_{t+1}^{(0)}+E_{t+1}^{(0)}}_{0}+\Vx[t+1]{}[(0)]).
\]
The following picture illustrates the basic policy value recursive formula in
multiple state model setting above.
\begin{center}
\begin{tikzpicture}
\draw[] (0,0) rectangle (2,1);
\node[] () at (1, 1.9) {(amount reserved \faIcon{at} time \(t\))};
\node[] () at (1, 1.4) {\(\Vx[t]{}[(j)]\)};
\draw[->, ForestGreen, line width=0.3mm] (0.6,-0.5) to (0.6,0.2);
\node[ForestGreen] () at (0.6,-0.8) {\(G_t^{(j)}\)};
\draw[->, red, line width=0.3mm] (1.4,0.2) to (1.4,-0.5);
\node[red] () at (1.4,-0.8) {\(e_t^{(j)}\)};
\node[draw] (start) at (1, -1.4) {State \(j\)};
\draw[-Latex, blue] (1,0.5) to[bend left] node[midway, auto]{\(\times (1+i)\)} (7,0.5);
\draw[] (4,-0.5) rectangle (10,1);
\node[] () at (11.5,0.25) {(what you have)};
\node[] () at (7,0.25) {\((\Vx[t]{}[(j)]+G_t^{(j)}-e_t^{(j)})(1+i)\)};
\node[rotate=90] () at (7,-1) {\faIcon{equals}};
\node[] () at (11.5,-3.5) {(what you need)};
\foreach \x in {0,...,3} {
\node[draw] () at (5, \x-5) {State \x};
\draw[-Latex] (start) --node[midway]{\(\px{x+t}[j\x]\)} (4.3,\x-5);
\node[] () at (8, \x-5) {\(B_{t+1}^{(\x)}+E_{t+1}^{(\x)}+\Vx[t+1]{}[(\x)]\)};
}
\draw[] (4,-6) rectangle (10,-1.5);
\end{tikzpicture}
\end{center}

\item \textbf{``NAAR'' form of basic policy value recursion.} Similar to
STAT3901, we can reorganize the terms in the basic policy value recursive
formula above to obtain its ``net amount at risk (NAAR)''\footnote{Strictly
speaking, in general multiple state model, the previous interpretation for NAAR
may not apply anymore, as we may not be considering ``dying''.} form:
\[
\overset{\text{(what you have)}}{\qty(\Vx[t]{}[(j)]+G_t^{(j)}-e_t^{(j)})(1+i)}
=\overset{\text{(what you need)}}{\Vx[t+1]{}[(j)]+B_{t+1}^{(j)}+E_{t+1}^{(j)}+
\sum_{k\ne j}^{}\px{x+t}[jk]
\underbrace{\qty(B_{t+1}^{(k)}+E_{t+1}^{(k)}+\Vx[t+1]{}[(k)]-\Vx[t+1]{}[(j)])}_{\text{``NAAR''}}},
\]
which follows from rewriting
\[
\px{x+t}[jj]\qty(\Vx[t+1]{}[(j)])
=\qty(1-\sum_{k\ne j}^{}\px{x+t}[jk])\qty(\Vx[t+1]{}[(j)])
=\Vx[t+1]{}[(j)]-\sum_{k\ne j}^{}\px{x+t}[jk]\qty(\Vx[t+1]{}[(j)])
\]
in the sum on the RHS of the original recursive formula.

The ``NAAR'' can be interpreted as the additional amount of money needed over
the reserve available for the case when the life goes to state \(j\)
(\(\Vx[t+1]{}[(j)]\)), in case the life goes to state \(k\ne j\) rather than
state \(j\) ``surprisingly''.

Thus, this recursive formula in ``NAAR'' form partitions ``what you need'' into
(i) amount needed (payment and reserve) when the life goes to state \(j\) and
(ii) extra amount needed over the reserve for going to state \(j\), as a
provision for the scenarios where the life goes to another state \(k\ne j\).

Again, this generalizes the NAAR form of basic policy recursive formula in
STAT3901, namely\\
\((\Vx[t]{}[]+G_t-e_t)(1+i)=\Vx[t+1]{}[]+\qx{x+t}(S_{t+1}+E_{t+1}-\Vx[t+1]{}[])\),
because we can express it as
\[
(\Vx[t]{}[(0)]+G_t^{(0)}-e_t^{(0)})(1+i)=\underbrace{B_{t+1}^{(0)}+E_{t+1}^{(0)}}_{0}+\Vx[t+1]{}[(0)]
+\px{x+t}[01](B_{t+1}^{(1)}+E_{t+1}^{(1)}+\underbrace{\Vx[t+1]{}[(1)]}_{0}-\Vx[t+1]{}[(0)])
\]
by considering the alive-dead model.

The following picture illustrates this recursive formula in ``NAAR'' form (with
\(j=3\)):
\begin{center}
\begin{tikzpicture}
\draw[] (0,0) rectangle (2,1);
\node[] () at (1, 1.4) {\(\Vx[t]{}[(3)]\)};
\draw[->, ForestGreen, line width=0.3mm] (0.6,-0.5) to (0.6,0.2);
\node[ForestGreen] () at (0.6,-0.8) {\(G_t^{(3)}\)};
\draw[->, red, line width=0.3mm] (1.4,0.2) to (1.4,-0.5);
\node[red] () at (1.4,-0.8) {\(e_t^{(3)}\)};
\node[draw] (start) at (1, -2) {State \(3\)};
\draw[-Latex, blue] (1,0.5) to[bend left] node[midway, auto]{\(\times (1+i)\)} (7,0.5);
\draw[] (4,-0.5) rectangle (10,1);
\node[] () at (11.5,0.25) {(what you have)};
\node[] () at (7,0.25) {\((\Vx[t]{}[(3)]+G_t^{(3)}-e_t^{(3)})(1+i)\)};
\node[rotate=90] () at (7,-1) {\faIcon{equals}};
\node[] () at (8,-7) {(what you need)};
\foreach \x in {0,...,2} {
\node[draw] () at (5, \x-5) {State \x};
\draw[-Latex] (start) --node[midway]{\(\px{x+t}[3\x]\)} (4.3,\x-5);
\node[] () at (9, \x-5) {\(B_{t+1}^{(\x)}+E_{t+1}^{(\x)}+\Vx[t+1]{}[(\x)]-\Vx[t+1]{}[(3)]\)};
}
\draw[] (4,-6) rectangle (12,-1.5);
\draw[] (4,-2.5) -- (12,-2.5);
\node[] () at (8,-2) {\(\Vx[t+1]{}[(3)]+B_{t+1}^{(3)}+E_{t+1}^{(3)}\)};
\draw[very thick, decorate,decoration={calligraphic brace, amplitude=5pt, raise=0pt}] (12.5,-2.5) -- (12.5,-6)
node[midway, right=0.2cm, text width=2cm]{extra for ``NAAR'' (expected)};
\end{tikzpicture}
\end{center}

\item \textbf{Continuous policy value recursion.} Again, for the sake of
simplicity, we will not deal with the \(1/m\)thly case here, though recursions
in such case can be developed in a similar way.  Nevertheless, here we will
discuss continuous policy value recursion in multiple state model (perhaps
more complex than the \(1/m\)thly case?  \faIcon[regular]{grin-beam-sweat}). We
have learnt the continuous policy value recursion in the alive-dead model in
\Cref{subsect:policy-val-cts-recurs}, which utilizes the \emph{Thiele's
differential equation}. So, as one may expect, here we are going to generalize
the Thiele's differential equation to the multiple state model setting.

Like what we did in \Cref{subsect:policy-val-cts-recurs}, we are going to study
continuous policy value recursion in three steps: (i) motivation, (ii) formula,
and (iii) intuitive interpretation.

\item \textbf{Motivation.} From the recursive formula in ``NAAR'' form above,
we can observe that changes in policy values for state \(j\) \(\Vx[t]{}[(j)]\)
are influenced by multiple factors: (i) \blc{interest}, (ii) \gc{premiums} \&
\rc{renewal expenses}, (iii) \vc{provision for ``NAAR''}, and \emph{(new!)}
(iv) \brc{payments \(B_{t+1}^{(j)}+E_{t+1}^{(j)}\) needed upon going to state
\(j\)}. So, as one may expect, these four factors would come into play in the
Thiele's differential equation.
\begin{center}
\begin{tikzpicture}
\draw[] (0,0) rectangle (2,1);
\node[ellipse, draw] () at (1, 1.4) {\(\Vx[t]{}[(3)]\)};
\draw[->, ForestGreen, line width=0.3mm] (0.6,-0.5) to (0.6,0.2);
\node[ForestGreen] () at (0.6,-0.8) {\(G_t^{(3)}\)};
\draw[->, red, line width=0.3mm] (1.4,0.2) to (1.4,-0.5);
\node[red] () at (1.4,-0.8) {\(e_t^{(3)}\)};
\node[draw] (start) at (1, -2) {State \(3\)};
\draw[-Latex, blue] (1,0.5) to[bend left] node[midway, auto]{\(\times (1+i)\)} (7,0.5);
\draw[] (4,-0.5) rectangle (10,1);
\node[] () at (11.5,0.25) {(what you have)};
\node[] () at (7,0.25) {\((\Vx[t]{}[(3)]+G_t^{(3)}-e_t^{(3)})(1+i)\)};
\node[rotate=90] () at (7,-1) {\faIcon{equals}};
\node[] () at (8,-7) {(what you need)};
\foreach \x in {0,...,2} {
\node[draw] () at (5, \x-5) {State \x};
\draw[-Latex] (start) --node[midway]{\(\px{x+t}[3\x]\)} (4.3,\x-5);
\node[violet] () at (9, \x-5) {\(B_{t+1}^{(\x)}+E_{t+1}^{(\x)}+\Vx[t+1]{}[(\x)]-\Vx[t+1]{}[(3)]\)};
}
\draw[] (4,-6) rectangle (12,-1.5);
\draw[] (4,-2.5) -- (12,-2.5);
\node[ellipse, draw] () at (6.5,-2) {\(\Vx[t+1]{}[(3)]\)};
\node[] () at (10.5,-2) {\brc{\(B_{t+1}^{(3)}+E_{t+1}^{(3)}\)}};
\draw[very thick, decorate,decoration={calligraphic brace, amplitude=5pt, raise=0pt}] (12.5,-2.5) -- (12.5,-6)
node[midway, right=0.2cm, text width=2cm, violet]{provision for ``NAAR''};
\draw[] (9,-1.5) -- (9,-2.5);
\end{tikzpicture}
\end{center}

\item\label{it:thiele-diff-eq-mult-state} \textbf{Thiele's differential
equation in multiple state model setting.} We shall consider continuous-time
multiple state model and fully continuous policies here. The \defn{Thiele's
differential equation} in multiple state model setting is given by
\[
\boxed{\dv{}{t}\Vx[t]{}[(j)]=\Vx[t]{}[(j)]\delta_t+G_t^{(j)}-e_{t}^{(j)}-R_t^{(j)}
-\sum_{k\ne j}^{}\qty(S_{t}^{(jk)}+E_{t}^{(jk)}+\Vx[t]{}[(k)]-\Vx[t]{}[(j)])\mu_{x+t}^{jk}}
\]
where:
\begin{itemize}
\item \(R_t^{(j)}\) is the time-\(t\) \underline{r}ate of \underline{annuity type}
benefit payments (with the associated expenses added, if any) while the life in
state \(j\).
\item \(G_t^{(j)}\) and \(e_t^{(j)}\) are the time-\(t\) rates of premium
payments and renewal expense payments while the life is in state \(j\), respectively.
\item \(S_t^{(jk)}\) and \(E_t^{(jk)}\) are the amounts of time-\(t\)
\underline{insurance type} benefit and the associated settlement expense, made
when the life enters state \(j\) \faIcon{at} time \(t\), respectively.
\end{itemize}
\begin{note}
Here we assume there is no initial expense.
\end{note}

\item \textbf{Intuitive interpretation.}
Like \Cref{subsect:policy-val-cts-recurs}, we integrate both sides from \(0\)
to a positive value \(s\):
\[
\Vx[s]{}[(j)]-\Vx[0]{}[(j)]=
\int_{0}^{s}\qty(\Vx[t]{}[(j)]\delta_t+G_t^{(j)}-e_{t}^{(j)}-R_t^{(j)}
-\sum_{k\ne j}^{}\qty(S_{t}^{(jk)}+E_{t}^{(jk)}+\Vx[t]{}[(k)]-\Vx[t]{}[(j)])\mu_{x+t}^{jk})
\dd{t}.
\]
Abusing notations slightly, we write
\[
\Vx[s]{}[(j)]=\Vx[0]{}[(j)]+
\int_{0}^{s}\blc{\Vx[t]{}[(j)]\delta_t\dd{t}}
+(\gc{G_t^{(j)}}-\rc{e_{t}^{(j)}})\dd{t}
-\brc{R_t^{(j)}\dd{t}}
-\vc{\sum_{k\ne j}^{}\qty(S_{t}^{(jk)}+E_{t}^{(jk)}+\Vx[t]{}[(k)]-\Vx[t]{}[(j)])\mu_{x+t}^{jk}\dd{t}}
.
\]
The expression in the integral incorporates the effects from all the four
factors discussed above. Let us elaborate more about the intuition:
\begin{enumerate}[label={(\arabic*)}]
\item Starting (time 0) reserve for state \(j\): \(\Vx[0]{}[(j)]\).
\item In every ``infinitesimal'' time interval \([t,t+\dd{t}]\) between time
\(0\) and time \(s\):
\begin{itemize}
\item \faIcon{plus-circle} \emph{(interest)} \(\blc{\Vx[t]{}[(j)]\delta_t\dd{t}}\)
is earned and added to the reserve for state \(j\).
\item \faIcon{plus-circle} \emph{(premium less expense)}
\((\gc{G_t^{(j)}}-\rc{e_t^{(j)}})\dd{t}\) is received and added to the reserve
for state \(j\).

\item \faIcon{minus-circle} \emph{(annuity benefit payment)}
\brc{\(R_t^{(j)}\dd{t}\)} is paid in the time interval \([t,t+\dd{t}]\) and
subtracted from the reserve for state \(j\).
\item \faIcon{minus-circle} \emph{(provision for ``NAAR'')} \\
\(\vc{\sum_{j\ne i}^{}\underbrace{\qty(S_{t}^{(jk)}+E_{t}^{(jk)}+\Vx[t]{}[(k)]-\Vx[t]{}[(j)])}_{\text{``NAAR''}}
\underbrace{\mu_{x+t}^{jk}\dd{t}}_{\mathclap{\text{``prob.''\ for \(j\to k\) in \([t,t+\dd{t}]\)}}}}\)
is subtracted from the reserve for state \(j\).
\end{itemize}
\end{enumerate}
\emph{(If you are interested)} \begin{remark}
\item The reason why only insurance type benefit ``\(S_t^{(jk)}\)'' appears in
the ``NAAR'', but not the total benefit \(B_{t}^{(k)}\), is that the amount of
annuity type benefit within the ``infinitesimal'' time interval
\([t,t+\dd{t}]\) is ``extremely small''. When multiplied by the also
``extremely small'' quantity \(\mu_{x+t}^{jk}\dd{t}\), its contribution would
be negligible.

\item The ``payments'' part (the fourth ``factor'') includes only the annuity
type benefit \brc{\(R_t^{(j)}\dd{t}\)}, without the insurance type benefit and
the associated settlement expense. This is because for the insurance type
benefit to be triggered in the ``infinitesimal'' time interval
\([t,t+\dd{t}]\), at least \emph{two} changes in states need to occur within
the interval: The life currently in state \(j\) needs to at least first go to
another state, and then enter back state \(j\). The chance of having such event
is ``negligible''\footnote{This is indeed related to an omitted assumption for
the continuous-time multiple state model. But this should be ``reasonable'' as
it is ``extremely unlikely'' for two changes in states to occur
``simultaneously'' (within the same ``infinitesimal'' time interval).}, so the
terms for insurance type benefit and its associated settlement expense can be
dropped.
\end{remark}
\item \textbf{Euler's method.} Again, we can apply the \emph{Euler's method} to
apply the Thiele's differential equation here in practice. Recall the formulas
are:
\begin{itemize}
\item \emph{Forward approximation:} Change \(\displaystyle
\dv{}{t}\Vx[t]{}[]\to\frac{\Vx[t+h]{}[]-\Vx[t]{}[]}{h}\).
\item \emph{Backward approximation:}  Change \(\displaystyle
\dv{}{t}\Vx[t]{}[]\to\frac{\Vx[t+h]{}[]-\Vx[t]{}[]}{h}\), \underline{and change
\(t\to t+h\) on the RHS} (quite many changes would be needed, be careful \warn{}).
\end{itemize}
\end{enumerate}
